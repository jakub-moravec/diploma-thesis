%%%%%%%%%%%%%%%%%%%%%%%%%%%%%%%%%%%%%%%%%%%%%%%%%
\section{Návrh}

%%%%%%%%%%%%%%%%%%%%%%%%%%%%%%%%%%%%%%%%%%%%%%%%%
% TODO vlastní vrstva řešící licence (licence na moduly vs licence na počty skriptů)

% Vrstvy:
% FilterLayer      - vrstva uplatňující filtry dle oprávnění (asi vhodné přejmenovat)
% Algorithm/Traversal layer - vrstva algoritmů a traversalů, resp. "business" v kontextu DAL, neměla by sahat přímo do DB. Tatvo vrstva by možná měla obsahovat ještě sourozence Index (lucene) a DAO (most na operations).
% "PredicateLayer" - vrstva provádějící drobný filtering výsledků z OperationsLayer (vypreparovat z operations). Je otázka, jestli je vrstva potřeba, některé filtry se totiž můžu naopak přesunout dolů, do Gremlin dotazů.
% OperationsLayer  - základní operace nad grafovou databází, jediná vrstva, která by měla přímo šahat do db (s výjimkou zanořování transakcí)
% ConnectionLayer  - konektor, pouze kód nutný k připojení do db a spouštění dotazů
% DB
%
% Model -stranou

% Rozdělení modulů na API a impl dle https://stackoverflow.com/questions/14856225/maven-separate-modules-for-interfaces-and-implementation-with-spring

% Konzistence
% - transakce deklarativní, u write operací mandatory propagation
%    - umožní občasné commity kvůli performance? - ANO, pomocnou metodou
%    - implementace pro Spring jsou v divném stavu, nebo nejsou, vadí to? - ???
%          - je potřeba zjistit, co by znamenalo implementovat pro gremlin 3, pokud by to bylo rozumné, není problém
%          - zjistit, jak udělat custom abstrakci transakcí
%          - zjistit, jak se dá customizovat rollback u deklarativních transakcí a jestli je to potřeba
% - zámky (long living transakce)
%    - Je potřeba řešit následující situace:
%       - vytváření nové verze (zámek na pracovní revizi)
%       - prune revizí
%       - dump revizí
%    a) AS IS (tzn read vs write na celou db)
%    b) zámky na jednotlivé revize (lock id v db na objektu revize)
% - merge bez VCS - v jakých případech? Je vlastní revize pro Merge? Pokud ne, jak to ovlivní výkon (je stále potřeba concurrent read-write)?

% performance
% - zámky na rodiče
% :: CLI běží typicky ve čtyřech vláknech, ta po skončení (zhruba stejný čas) provolávají server - merger. Ten tedy také běží paralelně
% ve čtyřech vláknech (vytváření objektů v db). V současné době zámek přidělující write transakce způsobuje, že jsou tato čtyři vlákna
% v podstatě serializována (respektive spíš doslova).
% Jednou z možností, jak toto vyřešit je navrhnout zámky o menší granularitě - na rodiče (tedy v tomto pořadí: super_root, resourcy, nody různých úrovní..). Tyto zámky by byly Javovské (pokud by se měnila architektura na cloud, tak by se musely propagovat do DB).
% :: Mechanismus pro uzly: při vytváření uzlu zámek na jeho rodiče. V rámci jedné transakce by takto byla vytvořená dávka uzlů e.g. 500, po commitu by byly všechny zámky uvolněny (zámky nemohou být uvolněny dříve, ostatní vlákna by neviděla výsledek - read commited).
% Pokud by mělo dojít k deadlockové situaci (vlákno má zámek na nějaké uzly, ale čeká na jiné zámky, které má jiné vlákno), provede se předčasný commit (zmenší se dávka) a tím se zámky uvolní.
% :: Mechanismus pro (flow) hrany: TODO
% Premisy:
%  - jak výchozí, tak koncový vrchol již nutně musí existovat (byly na začátku stejného CSV)
% Návrhy:
%  - mělo by stačit zamykat výchozí vrchol (hrany by byly vytvářeny jen ze směru z->do)
%  - kdyby u hran byl odkaz i na rodiče zdroje (např. na tabulku sloupečku), stačilo by zamknout tabulku a flow hrany pod tímto zámkem vložit všechny (TODO jsou hrany seřazeny po tabulkách, nebo náhodně?)
% read lock = nolock ?
% zámky na revize pro stop-the world?


%\subsection{Architektonická omezení}
% Podněty
% - Architektura vyhovující cloudovým požadavkům
%  - žádný filesystem, jen pipes na jiné služby, které mohou filesystem podporovat (lze využívat pouze temp uložiště a sfree buckety)
%  - např jedna služba client, jedna server, jedna grafovka
%  - jak by se řešily veškeré konfigurace

% Škálovatelnost - cloud
% https://acloud.guru/
% https://www.amazon.com/Patterns-Enterprise-Application-Architecture-Martin/dp/0321127420
% https://martinfowler.com/articles/microservices.html
% https://martinfowler.com/tags/application%20architecture.html - Serverless Architectures on AWS
% https://www.youtube.com/watch?v=LAWjdZYrUgI
% AWS Lambda - amazon SaaS (?)
% S3 - cloud storage (?)
% Cloud computing http://nvlpubs.nist.gov/nistpubs/Legacy/SP/nistspecialpublication800-145.pdf
% JPMC GAIA https://www.americanbanker.com/news/unexpected-champion-of-public-clouds-jpmorgan-cio-dana-deasy
% Cloud Awareness ;  OpenStack
