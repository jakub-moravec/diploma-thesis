%%%%%%%%%%%%%%%%%%%%%%%%%%%%%%%%%%%%%%%%%%%%%%%%%
\section{Návrh}

%%%%%%%%%%%%%%%%%%%%%%%%%%%%%%%%%%%%%%%%%%%%%%%%%
% TODO vlastní vrstva řešící licence (licence na moduly vs licence na počty skriptů)

% Vrstvy:
% FilterLayer      - vrstva uplatňující filtry dle oprávnění (asi vhodné přejmenovat)
% Algorithm/Traversal layer - vrstva algoritmů a traversalů, resp. "business" v kontextu DAL, neměla by sahat přímo do DB. Tatvo vrstva by možná měla obsahovat ještě sourozence Index (lucene) a DAO (most na operations).
% "PredicateLayer" - vrstva provádějící drobný filtering výsledků z OperationsLayer (vypreparovat z operations). Je otázka, jestli je vrstva potřeba, některé filtry se totiž můžu naopak přesunout dolů, do Gremlin dotazů.
% OperationsLayer  - základní operace nad grafovou databází, jediná vrstva, která by měla přímo šahat do db (s výjimkou zanořování transakcí)
% ConnectionLayer  - konektor, pouze kód nutný k připojení do db a spouštění dotazů
% DB
%
% Model -stranou

% Konzistence
% - transakce deklarativní, u write operací mandatory propagation
%    - umožní občasné commity kvůli performance?
%    - implementace pro Spring jsou v divném stavu, nebo nejsou, vadí to?
% - zámky (long living transakce)
%    a) AS IS (tzn read vs write na celou db)
%    b) zámky na jednotlivé revize (lock id v db na objektu revize)
