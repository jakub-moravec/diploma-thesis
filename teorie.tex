\chapter{Teoretická část}
%TODO write something nice

\section{Data Layer}

\section{Data(base) Layer Abstraction}
Abstrakce datové vrstvy (a databází) je historicky hojně diskutovaným tématem, %TODO zdroj
střetává se zde několik zájmů. Nejčastější argumenty proti tomuto principu míří na 
ztrátu výkonost při zařazování abstraktních vrstev. Tato ztráta výkonu je potom často způsobena 
právě úrovní abstrakce, konkrétně neschopností vývojáře použít konstrukty specifické pro dané databázové systémy. 
Je ale důležité si uvědomit, že odstranění vendor lockingu není jediným a ani primárním cílem \nom{DLA}{Database Layer Abstraction}.
Důležitějším cílem je zjednodušení aplikační logiky a její očištění o konstrukty z databázového světa. 

%TODO úrovně abstrakce viz http://blog.gauffin.org/2013/01/data-layer-the-right-way/ + dohledat lepší zdroj

%TODO data access alternatives - pokud to není to samé, ten samý zdroj

%TODO Data layer patterns -ten samý zdroj + https://thinkinginobjects.com/2012/08/26/dont-use-dao-use-repository/ or https://medium.com/@krzychukosobudzki/repository-design-pattern-bc490b256006

%==================================%
%              NOTES               %
%==================================%

% - domain and data mapping layer