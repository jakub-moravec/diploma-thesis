\chapter{Závěr}
Cílem této práce bylo navržení vícevrstvé architektury a příslušných \textit{API} pro práci s grafovou databází realizující metadatové úložiště aplikace \textit{Manta Flow}. Tato architektura byla navržena, byla vytvořena její prototypová implementace a ta byla otestována. Prototypová implementace navržené architektury zahrnující také dvě stěžejní operace byznys logiky aplikace prokazuje, že vytvořený návrh je pro práci s metadatovým úložištěm aplikace \textit{Manta Flow} vhodný. Navržená architektura striktně odděluje byznys logiku aplikace a perzistentní logiku pro ukládání metadat do grafové databáze pomocí vytvořeného \textit{API} perzistentní vrstvy, a řeší tak řadu současných architektonických omezení aplikace.

% modularity
Nejpodstatnějším důsledkem navržené architektury je umožnění změny grafové databáze (i změny jazyka dotazujícího grafovou databázi) realizující metadatové úložiště bez dopadů na vyšší vrstvy aplikace. Tento požadavek měl největší prioritu, protože aktuálně použiváná grafová databáze \textit{Titan} již není nadále vyvýjena a brzy nebude ani podporována. Zároveň dotazovací jazyk \textit{Gremlin 2.x}, který byl hojně používán i v implementacích byznys logiky aplikace nepodporuje všechny grafové databáze, které by mohly být potenciálně vhodné pro realizaci metadatového úložiště \textit{Manta Flow}.

% transactinos and domain model
Aby bylo zajištěno správné používání navrženého \textit{API}, musely být eliminovány možnosti jeho obcházení. Ty spočívaly především v používaném transakčním modelu, který umožňoval v kombinaci s používanými entitami doménového modelu \textit{TinkerPop Blueprints} přímé dotazy do grafové databáze z téměř jakékoliv části serverové části aplikace. Byl tak navržen nový doménový model aplikace a byl zaveden deklarativní transakční model. Ten zaručuje možnost propagace transakcí do vyšších vrstev aplikace bez toho, aby tak byl z těchto vrstev umožněn přímý přístup do grafové databáze. Pro použití deklarativního transakčního modelu musí být splněny (v závislosti na frameworku, který jej implementuje) jisté předpoklady. U grafových databází zatím neexistují standardní implementace používající deklarativní transakční model, byly tak vytvořeny dvě \textit{PoC} implementace ověřující možnost použití tohoto transakčního modelu v rámci frameworku \textit{Spring} a s využitím dotazovacích jazyků \textit{Gremlin 2.x} a \textit{Gremlin 3.x}. Obě implementace možnost použití deklarativního transakčního modelu potvrdily.

% scaling
Definované \textit{API} perzistentní vrstvy navržené architektury také mění granularitu dotazů do metadatového úložiště. Stávající implementace používá často atomické dotazy do grafové databáze a \textit{de-facto} tím kvůli minimalizaci režie dotazů vynucuje použití \textit{embedded} databáze jako podkladové vrstvy pro grafovou databázi \textit{Titan}. Navržené \textit{API} umožňuje používání komplexnějších dotazů do metadatového úložiště a v zásadě tak umožňuje použití vzdálené grafové databáze. Tím jsou otevřeny nové možnosti pro horizontální škálování aplikace. V práci je navrženo několik návazných úprav architektury aplikace pro dosažení horizontálního škálování, přičemž první a potenciálně nejefektivnější z nich je použití vzdálené horizontálně škálovatelné grafové databáze pro realizaci metadatového úložiště.

% orchestration
Součástí celého řešení \textit{Manta Flow} se také nově staly aplikace \textit{Configurator} a \textit{Updater}, které zatím ale nejsou ukotveny v architektuře aplikace, která se navíc mění výše popsaným způsobem. Bylo tak nutné vytvořit návrh orchestrace všech aplikací tvořících celé řešení \textit{Manta Flow} a to v případě nasazení všech aplikace na jedno zařízení, nebo na několik různých zařízení. Tento návrh má za následky další úpravy architektury aplikací \textit{Manta Flow Client, Updater a Configurator}. V práci jsou tyto návrhy úprav architektury jednotlivých aplikací popsány a zdůvodněny.

\section{Možnosti dalšího rozvoje}

V diplomové práci bylo navrženo několik změn architektury aplikace \textit{Manta Flow}. Pro část z nich, konkrétně pro vícevrstvou architekturu navrženou pro přístup do metadatového úložiště, byla také vytvořena prototypová implementace ověřující vhodnost vytvořeného návrhu. Nejdůležitější návaznou činností na tuto práci tak je posouzení návrhů na úpravy architektury aplikace a jejich případná implementace do produkční verze aplikace.

V průběhu tvorby diplomové práce bylo také identifikováno několik oblastí, které by mohly být dále zkoumány a rozvíjeny.

% zámky & paralelism
Kvůli neschopnosti aplikace provádět paralelní vkládání objektů do grafové databáze byl navržen nový algoritmus pro zamykání objektů v databázi. Díky němu je nyní teoreticky možné řádově zrychlit několikahodinový proces aktualizace metadatového úložiště. Je však nutné upravit algoritmy, které jsou součástí tohoto procesu, tak, aby byly schopné nový systém zámků efektivně využít.

% query a traverse metody
Součástí navrženého \textit{API} perzistentní vrstvy jsou \textit{query} metody, které slouží pro tvorbu komplexních dotazů do metadatového úložiště. Byly uvedeny argumenty, proč je výhodnější sestavování komplexních dotazů, které jsou jako celek vykonávány grafovou databází, než řetězení dotazů atomických. Mohlo by proto být výhodné rozhraní \textit{query} metod dále rozšířit, aktuálně totiž nepokrývá všechny typy dotazů. Analogicky by také mohl být definován nový typ metod \textit{API} - \textit{traverse}, který by pomocí obdobně strukturovaných dotazů prováděl průchody grafovou databází. Algoritmy, které jsou součástí byznys vrstvy aplikace by tak mohly provádět řádově méně dotazů do grafové databáze, čímž by bylo možné výrazně snížit režii těchto dotazů.

% microservices
V sekce návrhu architektury týkající se škálovatelnosti aplikace byla mimo jiné diskutována možnost úpravy navržené architektury postupnými kroky na architekturu \textit{microservices}, která by umožnila lepší horizontální škálování aplikace. Vzhledem k nevhodnosti této architektury pro aktuální model nasazení aplikace byla tato možnost v práci označena jako nevhodná pro aplikaci \textit{Manta Flow}. Je ale možné, že kvůli zvyšujícím se nárokům na objem zpracovávaných dat bude nutné ji v budoucnu opět zvážit.
