%%%%%%%%%%%%%%%%%%%%%%%%%%%%%%%%%%%%%%%%%
% Beamer Presentation
% LaTeX Template
% Version 1.0 (10/11/12)
%
% This template has been downloaded from:
% http://www.LaTeXTemplates.com
%
% License:
% CC BY-NC-SA 3.0 (http://creativecommons.org/licenses/by-nc-sa/3.0/)
%
%%%%%%%%%%%%%%%%%%%%%%%%%%%%%%%%%%%%%%%%%

%----------------------------------------------------------------------------------------
%	PACKAGES AND THEMES
%----------------------------------------------------------------------------------------

\documentclass{beamer}

\mode<presentation> {
\usetheme{Madrid}
\usecolortheme{dolphin}

\newcounter{saveenumi}
\newcommand{\seti}{\setcounter{saveenumi}{\value{enumi}}}
\newcommand{\conti}{\setcounter{enumi}{\value{saveenumi}}}


%\setbeamertemplate{footline} % To remove the footer line in all slides uncomment this line
%\setbeamertemplate{footline}[page number] % To replace the footer line in all slides with a simple slide count uncomment this line

%\setbeamertemplate{navigation symbols}{} % To remove the navigation symbols from the bottom of all slides uncomment this line
}

\usepackage{graphicx} % Allows including images
\usepackage{booktabs} % Allows the use of \toprule, \midrule and \bottomrule in tables

%----------------------------------------------------------------------------------------
%	TITLE PAGE
%----------------------------------------------------------------------------------------

\title[Diplomová práce]{Analýza a návrh abstraktní vícevrstvé architektury pro práci s grafovou databází realizující metadatové úložiště pro data lineage} % The short title appears at the bottom of every slide, the full title is only on the title page

\author{Bc. Jakub Moravec} % Your name
\institute[ČVUT] % Your institution as it will appear on the bottom of every slide, may be shorthand to save space
{
Vedoucí: Ing. Michal Valenta, Ph.D. \\
Oponent: Ing. Jiří Šebek \\
\medskip
České vysoké učení technické v Praze \\
Fakulta Elektrotechnická \\
Otevřená Informatika, Softwarové Inženýrství \\
}
\date{21.6.2018} % Date, can be changed to a custom date

\begin{document}

\begin{frame}
\titlepage
\end{frame}

\begin{frame}
\frametitle{Obsah}
\tableofcontents
\end{frame}

%----------------------------------------------------------------------------------------
%	PRESENTATION SLIDES
%----------------------------------------------------------------------------------------


\section{Uvedení kontextu}
\begin{frame}
\frametitle{Uvedení kontextu}
   \begin{itemize}
      \item
   \end{itemize}
\end{frame}

\section{Definice problému}
\begin{frame}
\frametitle{Definice problému}
   \begin{itemize}
      \item Existuje velké množství grafových databází
      \item Jednotlivé databáze se výrazně liší ve výkonostních parametrech
      \item Oblast je dinamická, vznikají nové nástroje, končí podpora pro některé stávající
      \item Neexistují obecné standardy pro jejich dotazování
      \item To vede k silné závislosti aplikací na používané grafové databázi a dotazovacím jazyce
      \begin{itemize}
         \item Dochází k prolínání perzistentní a byznys logiky aplikace
         \item Aplikace je obtížně spravovatelná a modifikovatelná
         \item Při zvolení nevhodné granularity dotazů je degradována efektivita grafové databáze
      \end{itemize}
   \end{itemize}
\end{frame}

\section{Cíle práce}
\begin{frame}
\frametitle{Cíle práce}
   \begin{itemize}
      \item Seznámení se s grafovými databázemi a jejich API
      \item Analýza způsobu využívání grafové databáze v aplikaci \textit{Manta Flow}
      \item Identifikace omezení stávající architektury aplikace vzhledem k práci s grafovou databází
      \item Rešerše existujících nástrojů pro abstrakci grafových databází
      \item Návrh vícevrstvé architektury abstrahující práci s grafovou databází
      \item Vytvoření prototypové implementace navržené architektury
   \end{itemize}
\end{frame}

\section{Dosažené výsledky}
\begin{frame}
\frametitle{Dosažené výsledky}
   \begin{enumerate}
      \item Rešerše grafových databází a možností jejich dotazování
      \item Rešerše softwarových architektur vhodných pro návrh architektury
      \item Analýza jednotlivých komponent aplikace \textit{Manta Flow} a způsobu práce aplikace s grafovou databází
      \item Identifikace omezení aplikace plynoucích ze stávající architektury aplikace
      \item Specifikace konkrétních požadavků na navrhovanou architekturu na základě analýzy
      \item Rešerše existujících nástrojů pro abstrakci grafových databází
      \seti
   \end{enumerate}
\end{frame}
\begin{frame}
\frametitle{Dosažené výsledky}
   \begin{enumerate}
      \conti
      \item Návrh vícevrstvé architektury pro práci s grafovou databází a příslušných API
      \item Prototypová implementace navržené architektury
      \item Otestování jednotlivých vrstev prototypové implementace pomocí jednotkových testů
      \item Implementace vybraných algoritmů tvořících byznys logiku aplikace pomocí prototypové implementace architektury
      \item Otestování implementovaných algoritmů pomocí jednotkových a integračních testů
      \item Návrh úpravy architektury dalších částí aplikace na základě navržené architektury pro práci s grafovou databází
   \end{enumerate}
\end{frame}


%------------------------------------------------
\begin{frame}
\frametitle{Návrh - vícevrstvá architektura}
   \begin{figure}
   \includegraphics[width=0.7\linewidth]{img/connector_modules}
   \end{figure}
\end{frame}
%------------------------------------------------

\section{Přínosy práce}
%------------------------------------------------
\begin{frame}
\frametitle{Přínosy práce}
% TODO
\begin{itemize}
\item Hlavní výstupy práce jsou:
   \begin{itemize}
      \item návrh vícevrstvév architektury,
      \item návrh příslušných API,
      \item prototypová implementace.
   \end{itemize}
\item Práce má přímé uplatnění ve splečnosti Manta.
\item Zadání práce bylo splněno.
\item Z posudků nevyplývají žádné otázky.
\end{itemize}
\end{frame}
%------------------------------------------------
\begin{frame}
   \centering\Large Děkuji za pozornost
   \vspace{2em}

   \centering\normalsize
   Jakub Moravec\\ jkb.moravec@gmail.com
\end{frame}


%------------------------------------------------

\begin{frame}
\frametitle{Blocks of Highlighted Text}
\begin{block}{Block 1}
Lorem ipsum dolor sit amet, consectetur adipiscing elit. Integer lectus nisl, ultricies in feugiat rutrum, porttitor sit amet augue. Aliquam ut tortor mauris. Sed volutpat ante purus, quis accumsan dolor.
\end{block}

\begin{block}{Block 2}
Pellentesque sed tellus purus. Class aptent taciti sociosqu ad litora torquent per conubia nostra, per inceptos himenaeos. Vestibulum quis magna at risus dictum tempor eu vitae velit.
\end{block}

\begin{block}{Block 3}
Suspendisse tincidunt sagittis gravida. Curabitur condimentum, enim sed venenatis rutrum, ipsum neque consectetur orci, sed blandit justo nisi ac lacus.
\end{block}
\end{frame}

%------------------------------------------------

\begin{frame}
\frametitle{Multiple Columns}
\begin{columns}[c] % The "c" option specifies centered vertical alignment while the "t" option is used for top vertical alignment

\column{.45\textwidth} % Left column and width
\textbf{Heading}
\begin{enumerate}
\item Statement
\item Explanation
\item Example
\end{enumerate}

\column{.5\textwidth} % Right column and width
Lorem ipsum dolor sit amet, consectetur adipiscing elit. Integer lectus nisl, ultricies in feugiat rutrum, porttitor sit amet augue. Aliquam ut tortor mauris. Sed volutpat ante purus, quis accumsan dolor.

\end{columns}
\end{frame}


\end{document}
