\chapter{Úvod}

\section{Data lineage}
% https://en.wikipedia.org/wiki/Data_lineage
% TODO - second turn

% Úvod
Data a z nich získávané informace vždy byly v centru pozornosti informačních technologií a jejich význam každým rokem stoupá. V digitální podobě jsou dnes zakódovány takřka všechny informace včetně našich 
osobních údajů, stavů účtů, plateb, atd. % TODO
A tyto informace jsou klíčové pro fungování firem i lidí. % TODO
Současně stále roste roste množství dat (zejména těch strojově generovaných). %TODO citace here
Je tedy kladena velká pozornost na procesy, kterými jsou data zpracovávána a pomocí kterých jsou z dat (ať už se jedná o Big Data, či "běžná data") získávány informace. Zde je známo mnoho zavedených i inovativních přístupů: \nomExpl{RDBS}{TODO} \nomExpl{OLTP}{TODO} databázeme, datové sklady, \nomExpl{OLAP}{TODO} analytické nástroje, (core) data miningové technologie, \nomExpl{NoSQL}{TODO} Big Data analytické nástroje, NewSQL nástroje. Ať už je zvolen kterýkoliv z těchto přístupů, procesy zpracovávající data bývají komplexní, často ne zcela intuitivní pro samotné vývojáři, natož potom pro analytiky či dokonce byznys uživatele. 

% Data Lineage
Do popředí se tak dostává nová skupina nástrojů označovaných jako Data lineage.\footnote{Stejně jako mnoho další termínů z oblasti informačních technologií se Data lineage nepřekládá, nebudeme ho tedy překládat ani my. Pokud bychom termín však přeci jen chtěli popsat českými slovy, nejvhodnější překlad by byl zřejmě "řízení datových toků".}
Jejich cílem je analyzovat (typicky) end-to-end datové toky systému - zdroje, transformace a cíle dat. To může být velmi komplexní úkol, informační systém se typicky skládá z řady navzájem propojených technologií\footnote{TODO příklady technology stacků}, a nástroj pro analýzu Data lineage si musí umět poradit nejen s každým z nich separátně, ale také s případnými transformacemi na hranicích těchto systémů. 
%TODO rozšířit

% Manta Flow
Jedním z úspěšných nástrojů pro Data lineage je Manta Flow\footnote{https://getmanta.com/}. Nástroj analyzuje zdrojové kódy vybraných RDBMS databází, Big Data nástrojů a ETL nástrojů. 
Zdrojové kódy analyzovaných systémů jsou pravidelně parsovány dle syntaktických a sémantických pravidel podporovaných nástrojů a následně jsou analyzovány přímé\footnote{TODO upřesnit} a nepřímé\footnote{TODO upřesnit} datové toky a transformace dat v informačním systému. Získané informace ukládány do metadatového uložiště, jímž je v současné době Titan ve verzi 0.4\footnote{TODO doplnit, odkaz}. Klientská část aplikace potom umožňuje uživateli vizualizovat datové toky dle zadaných parametrů (zdroj a cíl datového toku, úroveň abstrakce atd.). Dynamicky tak vznikají komplexní dotazy do metadatové databáze, pomocí kterých jsou procházeny grafy datových toků a vraceny výsledky.    

\section{Definice problému}
% Neexistence standardů GDB
Přestože má použití \nomExpl{GDB}{Grafová databáze} jako metadatového uložiště pro Data lineage nástroje silné opodstatnění (rozebráno v kapitole TODO), přináší s sebou krom nesporných výhod také řadu problémů. Jejich společným jmenovatelem je fakt, že koncept \nom{GDB} je (minimálně v širším pohledu) relativně nový\footnote{TODO detail a citace here} a stále prochází dynamickým rozvojem. Nejsou tak zatím definovány jasné standardy pro řadu funkcí GDB. Nejpalčivější je tanto problém v případě konektivity a dotazování GDB\footnote{TODO citace here}. Proto také v tuto chvíli neexistují široce uznávané \i{best practises} softwarového inženýrství pro tvorbu \nomExpl{DAL}{Database Abstraction Layer} při využití GDB. Není tak překvapaním, že je při používání GDB v aplikacích často míchána perzistentní a byznys logika aplikace\footnote{TODO citace here (snad se najde)} - což je typickou ukázkou špatného návrhu\footnote{TODO citace here}. Je tak nicméně činěno často zcela vědomně a to čistě z neexistence lepšího řešení. Pro produkt Manta Flow je tento problém velice aktuální - používaná GDB Titan 0.4 již není dále podporovaná a je tak velmi pravděpodobné, že dojde k její výměně za jinou technologii. Cílem této práce je navrhnout architekturu \nom{DAL}, která bude vyhovovat potřebám nástroje Manta Flow a bude v co největší míře oddělovat perzistentní logiku od zbytku aplikace. Zavedení této vrstvy aplikace bude pravděpodobně znamenat zásah do celé architektury aplikace. Součástí práce tedy musí být nový návrh architektury aplikace reflektující změnu v přístupu k GDB. 

\section{Struktura diplomové práce}
Práce je rozdělena na teoretickou a experimentální část. 

% Teorietická část
Cílem teoretické části je popsat obecné principy GDB, jejich vnitřní organizaci a možnosti dotazování (TODO kapitola). Dále jsou popsány možnosti abstrakce ve světě softwarového inženýrství - ať už na úrovní procesů a \nomExpl{API}{Aplikační programové rozhraní}, nebo na nižších úrovních - například návrhové vzory (TODO kapitola). 

%Experimentální část
Experimentální část práce obsahuje hlubší analýzu potřeb projektu Manta Flow na nové \nom{DAL} (TODO kapitola). Na základě teoretické části práce je potom navržena vícevrstvá abstraktní architektura \nom{DAL} (TODO kapitola). Na základě návrhu je vytvořeno \nomExpl{PoC}{Proof of Concept} řešní a to otestováno (TODO kapitola). 
% TODO doplnit