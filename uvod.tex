\chapter{Úvod}

\section{Data lineage}
% https://en.wikipedia.org/wiki/Data_lineage
% TODO - second turn

% Úvod TODO velký špatný
Data a z nich získávané informace vždy byly v centru pozornosti informačních technologií a jejich význam každým rokem stoupá. V digitální podobě jsou dnes zakódovány takřka všechny informace včetně našich osobních údajů, bankonvních transakcí, či zdravotních informací. 
Tyto informace jsou klíčové pro fungování firem i lidí. % TODO
Současně stále roste roste množství dat, zejména těch strojově generovaných . %TODO citace here
Je tedy kladena velká pozornost na procesy, kterými jsou data zpracovávána a pomocí kterých jsou z dat získávány informace. Zde je známo mnoho zavedených i inovativních přístupů: \nomExpl{RDBS}{TODO} \nomExpl{OLTP}{TODO} databázeme, datové sklady, \nomExpl{OLAP}{TODO} analytické nástroje, data miningové technologie, \nomExpl{NoSQL}{TODO} Big Data analytické nástroje, NewSQL nástroje. Ať už je zvolen kterýkoliv z těchto přístupů, procesy zpracovávající data bývají komplexní, často ne zcela intuitivní pro samotné vývojáři, natož potom pro analytiky či dokonce byznys uživatele. 

% Data Lineage
Do popředí se tak dostává nová skupina nástrojů označovaných jako Data lineage.\footnote{Stejně jako mnoho další termínů z oblasti informačních technologií se Data lineage nepřekládá, nebudeme ho tedy překládat ani my. Pokud bychom termín však přeci jen chtěli popsat českými slovy, nejvhodnější překlad by byl zřejmě "řízení datových toků".}
Jejich cílem je analyzovat end-to-end datové toky v systému - zdroje, transformace a cíle dat. To může být velmi komplexní úkol, informační systém se typicky skládá z řady navzájem propojených technologií, a nástroj pro analýzu Data lineage si musí umět poradit nejen s každým z nich separátně, ale také s případnými transformacemi na hranicích těchto systémů. 
%TODO rozšířit

% Manta Flow
Jedním z úspěšných nástrojů pro Data lineage je Manta Flow\footnote{https://getmanta.com/}. Nástroj analyzuje zdrojové kódy vybraných RDBMS databází, Big Data nástrojů a ETL nástrojů. 
Zdrojové kódy analyzovaných systémů jsou pravidelně parsovány dle syntaktických a sémantických pravidel podporovaných nástrojů a následně jsou analyzovány přímé a nepřímé\footnote{Představme si relační databázi s tabulkami A, B a C. Mezi tabulkami A a B bude přímý datový tok, tzn. data z tabulky A budou TODO} datové toky a transformace dat v informačním systému. Získané informace jsou ukládány do metadatového uložiště, jímž je v současné době grafová databáze Titan\footnote{http://titan.thinkaurelius.com/}. Klientská část aplikace potom umožňuje uživateli vizualizovat datové toky dle zadaných parametrů (zdroj a cíl datového toku, úroveň abstrakce atd.). Dynamicky tak vznikají komplexní dotazy do metadatové databáze, pomocí kterých jsou procházeny grafy datových toků a vraceny výsledky.    

\section{Definice problému}
% Neexistence standardů GDB
Přestože má použití grafové databáze jako metadatového uložiště pro Data lineage nástroje silné opodstatnění\footnote{Grafová databáze umožňuje výrazně rychlejší hledání datových toků v informačních systémech, než by umožňovali jiné architektury. Způsob procházení grafů je popsán v kapitole \ref{sec:gdb-dotazy}.}, přináší s sebou krom nesporných výhod také řadu problémů. Jejich společným jmenovatelem je fakt, že v oblasti grafových databází, která je relativně nová a stálo prochází dynamickým rozvojem, nejsou zatím jasně definovány obecně podporované standardy. Neexistuje například univerzální, stabilní a obecně podporovaný dotazovací jazyk pro grafové databáze (například v oblasti relačních databází tuto úlohu plní SQL). Proto také nejsou v tuto chvíli definovány doporučené postupy softwarového inženýrství pro tvorbu abstraktních rozhraní pracujících s grafovými databázemi. Není tak překvapaním, že je při používání grafových databází v aplikacích často míchána perzistentní a byznys logika aplikace\footnote{TODO citace here (snad se najde)} - což je typickou ukázkou špatného návrhu\footnote{TODO citace here}. Je tak nicméně činěno často zcela vědomně a to čistě z neexistence lepšího řešení. Pro produkt Manta Flow je tento problém velice aktuální - používaná databáze Titan 0.4 již není dále podporovaná \cite{Titan04} a je pravděpodobné, že dojde k její výměně za jinou technologii. Cílem této práce je navrhnout abstraktní architekturu pro práci s grafovou databází, která bude vyhovovat potřebám nástroje Manta Flow a bude v co největší míře oddělovat perzistentní logiku od zbytku aplikace. Zavedení této vrstvy aplikace bude pravděpodobně znamenat zásah do celé architektury aplikace. Součástí práce tedy musí být nový návrh architektury aplikace reflektující změnu v přístupu ke grafové databází. 

\section{Struktura diplomové práce}
Práce je rozdělena na teoretickou a experimentální část. 

% Teorietická část
Cílem teoretické části je popsat obecné principy grafových databází, jejich vnitřní organizaci a možnosti dotazování dat. Dále jsou popsány možnosti abstrakce ve světě softwarového inženýrství - ať už na úrovní procesů a programových rozhraní (API), nebo na nižších úrovních - například návrhových vzorů. 

%Experimentální část
Experimentální část práce obsahuje hlubší analýzu potřeb projektu Manta Flow pro přístup ke grafové databázi. Na základě teoretické části práce je navržena vícevrstvá abstraktní architektura, vytvořeno \nomExpl{PoC}{Proof of Concept} řešní a to otestováno nad reálnými data. 
