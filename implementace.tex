\chapter{Implementace prototypu}
Součástí práce je prototypová implementace architektury navržené v kapitole \ref{sec:des_api}, konkrétně komponent \textit{Doménový model, Databázová vrstva, Perzistentní vrstva a Vrstva datového přístupu}. Prototypová implementace také obsahuje část byznys logiky vyšších vrstev aplikace, na jejímž základě je provedena validace navržené architektury a navrženého \textit{API} perzistentní vrstvy.  Na obrázku \ref{fig:poc_components} je \textit{UML} diagram komponent prototypové implementace. V této kapitole jsou blíže popsány důležité či nestandardní části implementace a je diskutována validnost vytvořeného návrhu.

%Domain
\section{Doménový model}
Doménový model definuje typy entit, se kterými může aplikace pracovat a možné parametry těchto entit. Prakticky tak definuje také schéma (datový model) grafové databáze, přestože ta nutně nemusí koncept schémat podporovat. Hlavní entity doménového modelu jsou popsány \textit{\nomExpl{BDM}{Business Domain Model} UML diagramem} \ref{fig:impl-domain} (cílem diagramu je popsat entity a vztahy mezi nimy, ne konkrétní implementaci). Při porovnání s datovým modelem metadatového úložiště je patrné, že:

\begin{itemize}
   \item Pro každý typ uzlu, který je součástí datového modelu existuje ekvivalent v doménovém modelu (s adekvátně definovanými parametry). Výjimku tvoří umělé typy uzlů, tedy \textit{REVISION\_ROOT, SOURCE\_ROOT a SUPER\_ROOT}.
   \item Doménový model neobsahuje tzv. \textit{řídící hrany} datového modelu, tedy hrany, které tvoří hiearchickou strukturu uzlů. Jediným netechnickým parametrem těchto hran je interval platnosti uzlů. Ten je ale spíše vlastností uzlů samotných, nikoliv jejich řídících hran (jeho umístění na řídící hrany v datovém modelu je důsledek optimalizace výkonu aplikace). Nic tedy nebrání tomu, aby byl doménový model zjednodušen a tyto hrany z něj odebrány.
   \item Doménový model obsahuje ekvivalenty hran typu \textit{DIRECT, FILTER a MAPS\_TO} datového modelu. Tyto hrany mají vlastní interval platnosti, který je sice omezen intervaly platnosti uzlů, které spojují, ale může se od těchto intervalů lišit. Hrany také obsahují další parametry podstatné pro analýzu datových toků. Entita doménového modelu \textit{Flow} zahrnuje hrany datového modelu \textit{DIRECT i FILTER}.
\end{itemize}


% TODO Databázová vrstva


%Perzistence
\section{Perzistentní vrstva}
Implementace perzistentní vrstvy je tvořena především implementací \textit{repository} objektů definovaných v \textit{API} vrstvy (popsáno v kapitole \ref{sec:des_persistence}). Dokumentace celého API perzistentní vrstvy je dostupná na přiloženém CD (viz obsah CD v příloze \ref{apx:cd}).

\subsection{Vrstva \textit{mapperů}}
Modul obsahující implementaci perzistentní vrstvy obsahuje další vnitřní vrstvu tzv. \textit{mapperů} - objektů provádějících mapování vrcholů a hran grafové databáze (reprezentovaných \textit{Gremlin} třídami \textit{Vertex a Edge}) na objekty doménového modelu.  Pro účely prototypu byla zvolena implementace pro (aplikací aktuálně používaný) jazyk \textit{Gremlin 2.x}.

\subsection{Dotazy do metadatového úložiště}
Jednoduché dotazy jsou implementovány nativními dotazy jazyka \textit{Gremlin}. Příklad \ref{exa:mapper} ukazuje mapování entity \textit{Flow} pomocí jazyka \textit{Gremlin 2.x}.

\lstinputlisting[language=Java, caption=Mapování entity \textit{Flow} (implementace pomocí \textit{Gremlin 2.x}), label=exa:mapper]{code/clear_gremlin.java}

Pro větší efektivitu dotazů byly v některých případech nativní \textit{Gremlin} dotazy nahrazeny dotazy pomocí další součástí \textit{TinkerPop} projektu - \textit{Pipes}\footnote{\url{https://github.com/tinkerpop/pipes/wiki}}. \textit{Pipes} slouží právě jako nástroj pro komplexnější průchody grafem pro jazyk \textit{Gremlin 2.x}\footnote{\textit{Gremlin 3.x} nahrazuje \textit{Pipes} navazujícím řešením s názvem \textit{Graph Traversal}}. Průchodu grafu pomocí \textit{Pipes} je ukázán na příkladu \ref{exa:pipes}, kde uvedený kód nalzene instanci entity \textit{Node} podle kvalifikovaného jména (jedná se tedy o průchod víceúrovňovou hierchií grafu datových toků).

\lstinputlisting[language=Java, caption=Nalezení uzlu dle kvalifikovaného jména (implementace pomocí \textit{Pipes}), label=exa:pipes]{code/pipes.java}

\subsection{\textit{Fluent API} pro \textit{query} metody}
TODO


%Data-access
\section{Vrstva datového přístupu}
\textit{API} této vrstvy v zásadě kopíruje \textit{API} perzistentní vrstvy.  U metod, u kterých musí být ověřována práva uživatele na dotazovaná data, je ale navíc přidán parametr reprezentující strategii, pomocí které budou data ověřována. Asi tedy nepřekvapí, že je vrstva implementována pomocí návrhovému vzoru \textit{Strategy} - aktuálně existují dvě strategie (v budoucnu může být ale tento seznam rozšířen):

\begin{itemize}
   \item{\textit{Oprávnění na základě pohledů a rolí}}: Je zaveden nový termín \textit{pohled}. Ten reprezentuje část dat uložených v metadatovém úložišti - každý pohled se skládá ze sady zahrnutých a vyloučených objektů ve smyslů hiearchie grafu datových toků (entity \textit{Node} a \textit{Resource}). Pohledů může být teoreticky neomezené množství a mohou se navzájem překrývat. Uživatelům jsou pak přidělována oprávnění na jednotlivé pohledy na základě jejich \textit{LDAP\footnote{Lightwight Directory Access Protocol}} rolí.

   \item{\textit{Neomezená oprávnění}}: Existují operace, u kterých není za žádných okolností žádoucí oprávnění kontrolovat. Například kompletní exporty metadatového úložiště. Proto existuje strategie přidělující všem uživatelům neomezená oprávnění.
\end{itemize}

Pro implementaci strategie kontroly oprávnění na základě pohledů je navíc za účelem rozdělení zodpovědností použit návrhový vzor \textit{Chain of responsibility} (součástí implementace je také vlastní \textit{cache} a další logika). Tato implmenetace je pospána diagramem tříd \ref{fig:impl-permission}.


%validation
\section{Validace a testování}
Funkcionalita obou implementovaných vrstev byla pokryta jednotkovými testy. %TODO test coverage

V rámci validace provedené implementace a návrhu vícevrstvé architektury byly pomocí těchto vrstev implementovány některé operace vyšších vrstev, konkrétně:

\begin{itemize}
   \item{\textit{Merge}}: Byla implementována operace \textit{merge}, tedy stěžejní část operace updatu metadatového úložiště. Operace je blíže popsána v kapitole \ref{sec:ana_merger}.
   \item{\textit{Flow Algorithm}}: Byl také implementován základní algoritmus pro hledání datových toků. Tento algoritmus je součástí mnoha sloužitějších algoritmů, které \textit{Manta Flow} používá.
\end{itemize}

Pro oba zmíněné algoritmy je v rámci stávající implementace definována řada jednotkových testů. Základním předpokladem validace vytvořeného prototypu je, že podaří-li se pomocí nově navržené architektury implementovat obě tyto funkcionality a budou-li nadále úspěšně procházet všechny testy pokrývající tyto funkcionality, podařilo se navrhnout architekturu a implementovat prototyp vyhovující zadání práce.

%TODO integrační test

Další typy testů (např. \textit{performance} testy) nebyly provedeny, protože v kontextu této práce nedávají smysl. Cílem validace je ověřit vhodnost navržené architektury pro konkrétní funkcionality \textit{Manta Flow}. Vytvořená implementace (byť rozsáhlá - více než 22 tisíc \textit{LOC}) je navíc pouze prototypem, neobsahuje tak všechny optimalizace výkonu, které zahrnuje stávající aplikace.
