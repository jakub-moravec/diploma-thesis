\documentclass[11pt,twoside,a4paper]{book}
% definice dokumentu
\usepackage[czech, english]{babel}
\usepackage[T1]{fontenc} 				% pouzije EC fonty
\usepackage[utf8]{inputenc} 			% utf8 kódování vstupu
\usepackage[square, numbers]{natbib}	% sazba pouzite literatury
\usepackage{indentfirst} 				% 1. odstavec jako v cestine, pro práci v aj možno zakomentovat
\usepackage{fancyhdr}					% tisk hlaviček a patiček stránek
\usepackage{nomencl} 					% umožňuje snadno definovat zkratky a jejich seznam

%%%%%%%%%%%%%%%%%%%%%%%%%%%%%%%%%%%%%%%%%%%%%%%%%%%%%%%%%%%%%%%
% informace o práci
\newcommand\WorkTitle{Analýza a návrh abstraktní vícevrstvé architektury pro práci s grafovou databází realizující metadatové úložiště pro data lineage}
\newcommand\FirstandFamilyName{Bc. Jakub Moravec}
\newcommand\Supervisor{Ing. Michal Valenta, Ph.D.}

\newcommand\TypeOfWork{Diplomová práce}

\newcommand\StudProgram{Otevřená Informatika, Magisterský}	% program
\newcommand\StudBranch{Softwarové inženýrství}				% obor

%%%%%%%%%%%%%%%%%%%%%%%%%%%%%%%%%%%%%%%%%%%%%%%%%%%%%%%%%%%%%%%
% minimální importy
\usepackage{graphicx}					% pro vkládání obrázků
\usepackage{k336_thesis_macros} 		% specialni makra pro formatovani DP a BP
\usepackage[
pdftitle={\WorkTitle},				% nastaví v informacích o pdf název
pdfauthor={\FirstandFamilyName},	% nastaví v informacích o pdf autora
colorlinks=true,					% před tiskem doporučujeme nastavit na false, aby odkazy a url nebyly šedé při ČB tisku
breaklinks=true,
urlcolor=red,
citecolor=blue,
linkcolor=blue,
unicode=true,
]
{hyperref}								% pro zobrazování "prokliknutelných" linků

% rozšiřující importy
\usepackage{listings} 			%slouží pro tisk zdrojových kódů se syntax higlighting
\usepackage{algorithm}
\usepackage{algorithmicx} 		%slouží pro zápis algoritmů
\usepackage{algpseudocode} 		%slouží pro výpis pseudokódu
\usepackage{amsmath}            %vzorce
\usepackage[dvipsnames]{xcolor} % barvy

%%%%%%%%%%%%%%%%%%%%%%%%%%%%%%%%%%%%%%%%%%%%%%%%%%%%%%%%%%%%%%%
% příkazy šablony
\makenomenclature								% při překladu zajistí vytvoření pracovního souboru se seznamem zkratek

\let\oldUrl\url									% url adresy budou zobrazeny: <url>
\renewcommand\url[1]{<\texttt{\oldUrl{#1}}>}

%%%%%%%%%%%%%%%%%%%%%%%%%%%%%%%%%%%%%%%%%%%%%%%%%%%%%%%%%%%%%%%
% vaše vlastní příkazy
\newcommand*{\nomExpl}[2]{#2 (#1)\nomenclature{#1}{#2}} 	% usnadňuje zápis zkratek : Slova ke Zkrácení (SZ)
\newcommand*{\nom}[2]{#1\nomenclature{#1}{#2}} 			% usnadňuje zápis zkratek : SZ

%%%%%%%%%%%%%%%%%%%%%%%%%%%%%%%%%%%%%%%%%%%%%%%%%%%%%%%%%%%%%%%
% listings syntax highlighting

\renewcommand{\lstlistingname}{Příklad}% Listing -> Příklad
\renewcommand{\lstlistlistingname}{Seznam příkladů}% List of Listings -> List of Algorithms

\lstset{ %
  backgroundcolor=\color{white},   % choose the background color
  basicstyle=\footnotesize,        % size of fonts used for the code
  breaklines=true,                 % automatic line breaking only at whitespace
  captionpos=b,                    % sets the caption-position to bottom
  commentstyle=\color{OliveGreen},    % comment style
  escapeinside={\%*}{*)},          % if you want to add LaTeX within your code
  keywordstyle=\color{blue},       % keyword style
  stringstyle=\color{red},     % string literal style
  showstringspaces=false,
  frame=lines,
}

%%%%%%%%%%%%%%%%%%%%%%%%%%%%%%%%%%%%%%%%%%%%%%%%%%%%%%%%%%%%%%%
% vlastní dokument
%%%%%%%%%%%%%%%%%%%%%%%%%%%%%%%%%%%%%%%%%%%%%%%%%%%%%%%%%%%%%%%
\begin{document}

	%%%%%%%%%%%%%%%%%%%%%%%%%%
	% nastavení jazyka, kterým je práce psána
	\selectlanguage{czech}	% podle jazyka práce nastavte na [czech | english]
	\translate				% nastaví české nebo anglické popisy (např. katedra -> department); viz k336_thesis_macros

	%%%%%%%%%%%%%%%%%%%%%%%%%%
	% Poznamky ke kompletaci prace
	% Nasledujici pasaz uzavrenou v {} ve sve praci samozrejme
	% zakomentujte nebo odstrante.
	% Ve vysledne svazane praci bude nahrazena skutecnym
	% oficialnim zadanim vasi prace.
	{
	\pagenumbering{roman} \cleardoublepage \thispagestyle{empty}
	\chapter*{Na tomto místě bude oficiální zadání vaší práce}
	\begin{itemize}
		\item Toto zadání je podepsané děkanem a vedoucím katedry,
		\item musíte si ho vyzvednout na studijním oddělení Katedry počítačů na Karlově náměstí,
		\item v jedné odevzdané práci bude originál tohoto zadání (originál zůstává po obhajobě na katedře),
		\item ve druhé bude na stejném místě neověřená kopie tohoto dokumentu (tato se vám vrátí po obhajobě).
	\end{itemize}
	\newpage
	}

	%%%%%%%%%%%%%%%%%%%%%%%%%%
	% Titulni stranka / Title page
	\coverpagestarts

	%%%%%%%%%%%%%%%%%%%%%%%%%%%
	% Poděkovani / Acknowledgements

	\acknowledgements
	\noindent
   Rád bych poděkoval Ing.~Michalu~Valentovi~Ph.D. za vedení diplomové práce a ochotu při konzultacích
   práce, které byly důležité pro správné směřování a úspěšné dokončení práce. Dále bych chtěl poděkovat
   RNDr.~Lukáši~Hermannovi a Ing.~Oldřichu~Nouzovi~Ph.D. za uvedení do kontextu aplikace \textit{Manta Flow} a trpělivé
   zodpovídání mnoha technických dotazů. Velké poděkování patří také mé rodině a přátelům
   za podporu při psaní diplomové práce.


	%%%%%%%%%%%%%%%%%%%%%%%%%%%
	% Prohlášení / Declaration
	\declaration{V Praze dne \today}

	%%%%%%%%%%%%%%%%%%%%%%%%%%%%
	% Abstrakt / Abstract

	\abstractpage

   Graph databases are a new and emerging technology, which results in absence of widely supported standards for their querying. Application \textit{Manta Flow}, which uses a graph database, faces its dependence on a concrete graph database and querying language, whose are both not developed any more and will not be supported soon.

   The objective of this thesis is to design and create prototype implementation of layered architecture abstracting the persistence logic of application \textit{Manta Flow} used for handling the graph database.
   The thesis contains research of graph databases, different levels of software abstractions, analysis of application \textit{Manta Flow}, its requirements and constrains for handling the graph database. On that basis layered architecture for handling graph database, an \textit{API} defining methods for querying the database and other related architectural changes are designed. The designed architecture is declared as a suitable solution of described problem based on created prototype implementation, its validation and testing.

	\vglue30mm

	\noindent{\Huge \textbf{Abstrakt}}
	\vskip 2.75\baselineskip

	Grafové databáze jsou novou rozvíjející se technologií a neexistují proto zatím obecně podporované standardy pro jejich dotazování. Aplikace \textit{Manta Flow}, která grafovou databází používá, se potýká se závislostí na konkrétní grafové databázi a dotazovacím jazyku, přičemž oba nástroje již nejsou vyvíjeny a brzy nebudou ani podporovány.

   Tato práce má za cíl vytvoření návrhu a prototypové implementace vícevrstvé architektury abstrahující perzistentní logiku aplikace \textit{Manta Flow} pro práci s grafovou databází.
   Součástí práce je rešerše grafových databází, různých úrovní softwarových abstrakcí, analýza aplikace \textit{Manta Flow}, jejích požadavků a omezení ve spojení s používáním grafové databáze. Na jejich základě je navržena vícevrstvá architektura pro práci s grafovou databází, \textit{API} definující metody pro přístup do databáze a další úpravy architektury aplikace. Je vytvořena prototypová implementace navržené architektury a ta je validována a testována, na základě čehož je navržená architektura označena jako vhodné řešení pro definovaný problém.


	%%%%%%%%%%%%%%%%%%%%%%%%%%
	% obsahy a seznamy
	\tableofcontents		% Obsah / Table of Contents

	% pokud v práci nejsou obrázky nebo tabulky - odstraňte jejich seznam
	\listoffigures			% Obsah / Table of Contents
	\listoftables			% Seznam tabulek / List of Tables
	\lstlistoflistings         % Seznam kódů

	%%%%%%%%%%%%%%%%%%%%%%%%%%
	% začátek textu
	\mainbodystarts

\chapter{Úvod}

\section{Data lineage}
% https://en.wikipedia.org/wiki/Data_lineage
% TODO - second turn

% Úvod TODO velký špatný
Data a z nich získávané informace vždy byly v centru pozornosti informačních technologií a jejich význam každým rokem stoupá. V digitální podobě jsou dnes zakódovány takřka všechny informace včetně našich osobních údajů, bankonvních transakcí, či zdravotních informací. 
Tyto informace jsou klíčové pro fungování firem i lidí. % TODO
Současně stále roste roste množství dat, zejména těch strojově generovaných . %TODO citace here
Je tedy kladena velká pozornost na procesy, kterými jsou data zpracovávána a pomocí kterých jsou z dat získávány informace. Zde je známo mnoho zavedených i inovativních přístupů: \nomExpl{RDBS}{TODO} \nomExpl{OLTP}{TODO} databázeme, datové sklady, \nomExpl{OLAP}{TODO} analytické nástroje, data miningové technologie, \nomExpl{NoSQL}{TODO} Big Data analytické nástroje, NewSQL nástroje. Ať už je zvolen kterýkoliv z těchto přístupů, procesy zpracovávající data bývají komplexní, často ne zcela intuitivní pro samotné vývojáři, natož potom pro analytiky či dokonce byznys uživatele. 

% Data Lineage
Do popředí se tak dostává nová skupina nástrojů označovaných jako Data lineage.\footnote{Stejně jako mnoho další termínů z oblasti informačních technologií se Data lineage nepřekládá, nebudeme ho tedy překládat ani my. Pokud bychom termín však přeci jen chtěli popsat českými slovy, nejvhodnější překlad by byl zřejmě "řízení datových toků".}
Jejich cílem je analyzovat end-to-end datové toky v systému - zdroje, transformace a cíle dat. To může být velmi komplexní úkol, informační systém se typicky skládá z řady navzájem propojených technologií, a nástroj pro analýzu Data lineage si musí umět poradit nejen s každým z nich separátně, ale také s případnými transformacemi na hranicích těchto systémů. 
%TODO rozšířit

% Manta Flow
Jedním z úspěšných nástrojů pro Data lineage je Manta Flow\footnote{https://getmanta.com/}. Nástroj analyzuje zdrojové kódy vybraných RDBMS databází, Big Data nástrojů a ETL nástrojů. 
Zdrojové kódy analyzovaných systémů jsou pravidelně parsovány dle syntaktických a sémantických pravidel podporovaných nástrojů a následně jsou analyzovány přímé a nepřímé\footnote{Představme si relační databázi s tabulkami A, B a C. Mezi tabulkami A a B bude přímý datový tok, tzn. data z tabulky A budou TODO} datové toky a transformace dat v informačním systému. Získané informace jsou ukládány do metadatového uložiště, jímž je v současné době grafová databáze Titan\footnote{http://titan.thinkaurelius.com/}. Klientská část aplikace potom umožňuje uživateli vizualizovat datové toky dle zadaných parametrů (zdroj a cíl datového toku, úroveň abstrakce atd.). Dynamicky tak vznikají komplexní dotazy do metadatové databáze, pomocí kterých jsou procházeny grafy datových toků a vraceny výsledky.    

\section{Definice problému}
% Neexistence standardů GDB
Přestože má použití grafové databáze jako metadatového uložiště pro Data lineage nástroje silné opodstatnění\footnote{Grafová databáze umožňuje výrazně rychlejší hledání datových toků v informačních systémech, než by umožňovali jiné architektury. Způsob procházení grafů je popsán v kapitole \ref{sec:gdb-dotazy}.}, přináší s sebou krom nesporných výhod také řadu problémů. Jejich společným jmenovatelem je fakt, že v oblasti grafových databází, která je relativně nová a stálo prochází dynamickým rozvojem, nejsou zatím jasně definovány obecně podporované standardy. Neexistuje například univerzální, stabilní a obecně podporovaný dotazovací jazyk pro grafové databáze (například v oblasti relačních databází tuto úlohu plní SQL). Proto také nejsou v tuto chvíli definovány doporučené postupy softwarového inženýrství pro tvorbu abstraktních rozhraní pracujících s grafovými databázemi. Není tak překvapaním, že je při používání grafových databází v aplikacích často míchána perzistentní a byznys logika aplikace\footnote{TODO citace here (snad se najde)} - což je typickou ukázkou špatného návrhu\footnote{TODO citace here}. Je tak nicméně činěno často zcela vědomně a to čistě z neexistence lepšího řešení. Pro produkt Manta Flow je tento problém velice aktuální - používaná databáze Titan 0.4 již není dále podporovaná \cite{Titan04} a je pravděpodobné, že dojde k její výměně za jinou technologii. Cílem této práce je navrhnout abstraktní architekturu pro práci s grafovou databází, která bude vyhovovat potřebám nástroje Manta Flow a bude v co největší míře oddělovat perzistentní logiku od zbytku aplikace. Zavedení této vrstvy aplikace bude pravděpodobně znamenat zásah do celé architektury aplikace. Součástí práce tedy musí být nový návrh architektury aplikace reflektující změnu v přístupu ke grafové databází. 

\section{Struktura diplomové práce}
Práce je rozdělena na teoretickou a experimentální část. 

% Teorietická část
Cílem teoretické části je popsat obecné principy grafových databází, jejich vnitřní organizaci a možnosti dotazování dat. Dále jsou popsány možnosti abstrakce ve světě softwarového inženýrství - ať už na úrovní procesů a programových rozhraní (API), nebo na nižších úrovních - například návrhových vzorů. 

%Experimentální část
Experimentální část práce obsahuje hlubší analýzu potřeb projektu Manta Flow pro přístup ke grafové databázi. Na základě teoretické části práce je navržena vícevrstvá abstraktní architektura, vytvořeno \nomExpl{PoC}{Proof of Concept} řešní a to otestováno nad reálnými data. 

\chapter{Teoretická část}
%TODO write something nice

%%%%%%%%%%%%%%%%%%%%%%%%%%%%%%%%%%%%%%%%%%%%%%%%%
\section{Grafové databáze}
%todo

%%%%%%%%%%%%%%%%%%%%%%%%%%%%%%%%%%%%%%%%%%%%%%%%%
\section{Architektury a API} %todo maybe better heading
% https://cw.fel.cvut.cz/wiki/courses/a4m36aos/useful_materials

%todo RPC

%todo CORBA

%todo Java RMI

%todo mention HTTP 1.0 specification 1996

%todo mention XML specification 1996

%todo XML-RPC 
\subsection{XML-RPC}
V roce 1998 se objevil soubor pravidel, který popisuje jak využívat technilogie RPC, XML a HTTP k vzdálenému volání procedur přes internet - \nom{XML-RPC}{Remote Procedure Calling protokol používající XML pro komuniakci přes internet}.\cite{Winner99} XML poskytuje slovník pro popis RPC volání přenášená mezi počítači pomocí HTTP protokolu (ačkoliv je popsán také obecný přístup nezávislý na konkrétním protokolu). Uživatel pošle požadavek na server implementující protokol. Uživatelem je typicky software volající konkrétní proceduru vzdáleného systému. Volání může obasahovat více vstupních parametrů a očekává jedny výstupní hodnotu. Vstupní parametry mohou být vnořené do kolekcí jako jsou mapy a listy, mohou být tedy posílány rozsáhlé struktury. Proto je možné používat XML-RPC pro posílání objektů a struktur (jako vstupních i výsupních parametrů). Díky možnosti popsání předávaných informací pomocí XML umožňuje XML-RPC prakticky vytváření \nomExpl{API}{aplikačních rozhraní} a tím komunikaci dvou a více nehomogeních prostředí.\cite{Laurent01}  
% Because XML-RPC is layered on top of HTTP, it inherits the inefficiencies of HTTP. This
% does place some limitations on its use in large-scale, high-speed applications, but inefficiency
% isn't important in many places. Although there are definitely high-profile projects for which
% systems must scale to millions of transactions at a time, keeping response time to a minimum,
% there are also many projects to which systems need to send information or request
% processing far less often -- from once a second to once a week -- and for which response
% time isn't absolutely critical. For these cases, XML-RPC can simplify developers' lives
% tremendously. 

%todo SOAP, maybe UDDI and WSDL, maybe JAX-RPC
% https://en.wikipedia.org/wiki/SOAP
% https://www.w3.org/TR/2000/NOTE-SOAP-20000508/
% https://www.w3.org/TR/soap/

%todo REST
% http://roy.gbiv.com/talks/200709_fielding_rest.pdf
% fielding disertation

%todo HATEOS

%%%%%%%%%%%%%%%%%%%%%%%%%%%%%%%%%%%%%%%%%%%%%%%%%
\section{Data Layer}

\section{Data(base) Layer Abstraction}
Abstrakce datové vrstvy (a databází) je historicky hojně diskutovaným tématem, %TODO zdroj
střetává se zde několik zájmů. Nejčastější argumenty proti tomuto principu míří na 
ztrátu výkonost při zařazování abstraktních vrstev. Tato ztráta výkonu je potom často způsobena 
právě úrovní abstrakce, konkrétně neschopností vývojáře použít konstrukty specifické pro dané databázové systémy. 
Je ale důležité si uvědomit, že odstranění vendor lockingu není jediným a ani primárním cílem \nom{DLA}{Database Layer Abstraction}.
Důležitějším cílem je zjednodušení aplikační logiky a její očištění o konstrukty z databázového světa. 

%TODO úrovně abstrakce viz http://blog.gauffin.org/2013/01/data-layer-the-right-way/ + dohledat lepší zdroj

%TODO data access alternatives - pokud to není to samé, ten samý zdroj

%TODO Data layer patterns -ten samý zdroj + https://thinkinginobjects.com/2012/08/26/dont-use-dao-use-repository/ or https://medium.com/@krzychukosobudzki/repository-design-pattern-bc490b256006 
% and https://martinfowler.com/eaaCatalog/repository.html (!)

%==================================%
%              NOTES               %
%==================================%

% - domain and data mapping layer
\chapter{Praktická část}

%%%%%%%%%%%%%%%%%%%%%%%%%%%%%%%%%%%%%%%%%%%%%%%%%
\section{Analýza}
% popis Manty
\subsection{Manta Flow}
\textit{Manta Flow}\footnote{https://getmanta.com/} je nástroj umožňující automatickou analýzu zdrojového kódu (SQL, Java) a následný popis transformační logiky v něm obsažené. Software je schopný rozpoznat i těžce čitelné konstruky zdrojového kódu. Díky tomu dokáže automaticky zanalyzovat rozsáhlé databáze a vytvořit z nich přehlednou mapu datových toků, neboli \textit{Data Lineage}. To se v praxi využívá převážně k optimalizaci datových skladů, snižování nákladů na vývoj softwaru, provádění dopadových analýz a při dokumentování prostředí pro potřeby regulačních úřadů.

Nástroj má dvě hlavní komponenty (viz diagram \ref{fig:ana-flow-comp}):
\begin{itemize}
	\item{\textit{Manta Flow CLI}}: je Java řádková aplikace provádějící extrakci skriptů ze zdrojových databází a uložišť a jejich analýzu. Analyzovaná data jsou následně poslána \textit{Manta Flow Serveru}. \textit{Klientská aplikace} také může nahrávat vygenerované exporty ze \textit{serveru} do externí metadatové databáze.
	\item{\textit{Manta Flow Server}}: je serverová Java aplikace, která ukládá získané informace do interního metadatového uložiště, transformuje je, umožňuje jejich visualizaci a přístup k nim pomocí veřejného API.
\end{itemize}

Interakce mezi \textit{klientskou} a \textit{serverovou} částí aplikace je popsána zjednodušeným sekvenčním diagramem \ref{fig:ana-flow-seq}. Pro tuto práci je podstatná především \textit{serverová} část aplikace a její interakce s metadatovým uložištěm (kapitola \ref{sec:ana_components}). \textit{Klientská} část aplikace je proto popsána méně detailně (kapitola \ref{sec:ana_other}).

\begin{figure}
\begin{center}
\includegraphics[width=14cm]{figures/flow_comp}
\caption{Architektura \textit{Manta Flow}}
\label{fig:ana-flow-comp}
\end{center}
\end{figure}

\begin{figure}
\begin{center}
\includegraphics[width=14cm]{figures/flow_seq}
\caption{Interakce mezi \textit{klientskou} a \textit{serverovou} částí \textit{Manta Flow}}
\label{fig:ana-flow-seq}
\end{center}
\end{figure}


\subsection{Metadatové uložiště}
\label{sec:ana_model}
Jak již bylo zmíněno v úvodu práce (kapitola \ref{sec:uvod}) metadatové uložiště produktu \textit{Manta Flow} je aktuálně implementováno grafovou databází \textit{Titan} (ve verzi 0.4) a je snaha o výměnu této databáze\cite{Kovar18}.
Než přistoupíme k bližšímu popisu jednotlivých komponent aplikace a jejich interakcí s metadatovým uložištěm (kapitola \ref{sec:ana_components}, je třeba nejdříve popsat entity, které jsou součástí analýzy datových toků a datový model metadatového uložiště (zobrazený na obrázku \ref{fig:ana-model}\footnote{Z modelu metadatového uložiště je zřejmé, že ne všechny podgrafy tvoří \textit{strom}. Vlastnosti \textit{stromu} nicméně porušují pouze hrany typu \textit{directFlow} a \textit{filterFlow}, které jsou výsledkem analýzy datových toků a jejichž odstraněním by strom vznikl. V textu je tak v některých případech používána teminologie vztahující se ke \textit{stromům} (například \textit{kořen}) - na celý graf je nahlíženo jako ne \textit{les}.}).

% entity datového modelu
V procesu analýzy datových toků hraje roli mnoho entit z analyzovaných systémů. Ty se navíc mohou výrazně lišit systém od systému - \textit{Manta Flow} může analyzovat širokou škálu spolu propojených databázových systémů a integračních služeb. Obecně lze říci, že každý systém obsahuje zdroje a cíle dat (tabulky, soubory, ...) a transformace dat (skripty, \textit{\nomExpl{ETL}{Extract Load Transform}} workflow, procedury, makra a další).

\begin{figure}
\begin{center}
\includegraphics[width=14cm]{figures/model}
\caption{Model grafové databáze}
\label{fig:ana-model}
\end{center}
\end{figure}

% datový model
Samotný datový model se skládá z devíti typů uzlů:

\begin{itemize}
	\item{\textit{SUPER\_ROOT}}: Uzel (právě jeden v databázi), který slouží jako umělý kořen všech uzlů typu \textit{RESOURCE}.
	\item{\textit{RESOURCE}}: Uzly tohoto typu reprezentují zdrojové systémy - zdroje definic objektů, zdrojových kódů, ETL řešení a další.
	\item{\textit{NODE}}: Uzly typu \textit{NODE} představují reálné objekty zdrojového systému - databáze, tabulky, sloupce, procedury, skripty a další.
	\item{\textit{LAYER}}: Uzly typu \textit{LAYER} reprezentují vrstvy modelu metadat. Datové toky nalezené při analýze zdrojových kódů jsou vždy ukládány do \textit{fyzické vrstvy}, ze které je potom možné generovat abstraktnější vrstvy modelu datových toků.
	\item{\textit{ATTRIBUTE}}: Uzly typu \textit{ATTRIBUTE} reprezentují atributy uzlů typu \textit{NODE} - parametry sloupců, popisy databázových objektů a další.
	\item{\textit{SOURCE\_ROOT}}: Uzel (právě jeden v databázi), který slouží jako umělý kořen všech uzlů typu \textit{SOURCE\_NODE}.
	\item{\textit{SOURCE\_NODE}}: Uzly typu \textit{SOURCE\_NODE} reprezentují soubory se zdrojovými kódy extrahovanými ze zdrojových systémů.
	\item{\textit{REVISION\_ROOT}}: Uzel (právě jeden v databázi), který slouží jako umělý kořen všech uzlů typu \textit{REVISION\_NODE}.
	\item{\textit{REVISION\_NODE}}: Uzly typu \textit{ATTRIBUTE} reprezentují revize modelu metadat, definují tedy jeho verzování. Kromě dalších parametrů mají všechny hrany grafu parametry \textit{tranEnd} a \textit{tranStart} definující platnost hran (viz obrázek \ref{fig:ana-model-rev}). Při každé analýze zdrojových systémů (která je prováděna dávkově klientskou částí aplikace) je vytvořena nová revize metadatového uložiště obsahující všechny objekty zdrojových systémů.\footnote{Je snaha tento princip upravit tak, aby byly objekty v metadatovém uložišti minimálně repklikovány \cite{Sykora17}.}
\end{itemize}

 a osmi typů hran:

 \begin{itemize}
	\item{\textit{hasResource}}: Hrana přiřazuje objekty (uzly typu \textit{NODE}) ke svým zdrojovým systémům (uzlům typu \textit{RESOURCE}). Hrana je také použite k propojení uzlů typu \textit{RESOURCE} s uzlem \textit{RESOURCE\_ROOT}.
	\item{\textit{hasParent}}: Hrana mezi dvěmi uzly typu \textit{NODE} vytvářející klasickou hiearchickou strukturu mezi těmito uzly - strom závislostí objektů zdrojových systémů.
	\item{\textit{directFlow}}: Hrana mezi dvěmi uzly typu \textit{NODE} říkající, že mezi těmito uzly existuje přímý datový tok (ve směru hrany).
	\item{\textit{filterFlow}}: Hrana mezi dvěmi uzly typu \textit{NODE} říkající, že mezi těmito uzly existuje nepřímý datový tok (ve směru hrany).
	\item{\textit{hasAttribute}}: Hrana přiřazující uzlům typu \textit{NODE} jejich atributy (uzly typu \textit{ATTRIBUTE}).
	\item{\textit{inLayer}}: Hrana typu \textit{inLayer} spojeju zdroje (uzly typu \textit{RESOURCE}) a vrstvy a říká, že zdroj patří do dané vrstvy modelu metadat.
	\item{\textit{hasSource}}: Hrana je použita k propojení uzlů reprezentujících zdrojové kódy (uzly typu \textit{SOURCE\_NODE}) s uzlem \textit{SOURCE\_ROOT}.
	\item{\textit{hasRevision}}: Hrana je použita k propojení uzlů reprezentujících revize modelu metadat (uzly typu \textit{REVISION\_NODE}) s uzlem \textit{REVISION\_ROOT}.
\end{itemize}

\begin{figure}
\begin{center}
\includegraphics[width=6cm]{figures/model_revisions}
\caption{Způsob verzování modelu metadat}
\label{fig:ana-model-rev}
\end{center}
\end{figure}

Uzly i hrany mají dle svého typu několik specifických atributů, ty ale nebudeme blíže popisovat, protože nejsou pro analýzu zásadní.

V metadatovém uložišti je dále definováno několik typů indexů, konkrétně se jedná o standardní indexy:

\begin{itemize}
	\item{\textit{indexy na kořeny}}: indexy pro konkrétní uzly, kořeny jednotlivých stromů datového modelu - \textit{SUPER\_ROOT, SOURCE\_ROOT, REVISION\_ROOT}
	\item{\textit{indexy na atributy hran}}: indexy zrychlující dohledávání atributů hran
	\item{\textit{indexy na typy hran}}: indexy zrychlující dohledávání hran daného typu pro jednotlivé uzly
\end{itemize}

a externí \textit{Apache Lucene} indexy, které slouží pro \textit{fulltextové} vyhledávání uzlů dle jejich názvů a pro intervalové vyhledání revizí.


\subsection{Popis komponent serverové části}
\label{sec:ana_components}

Aplikace \textit{Manta Flow} praceje s metadatovým uložištěm několika různými způsoby, přičemž různé moduly aplikace využívají jeden či více těchto přístupů (a často také provádí vlastní pomocné dotazy přímo do metadatového uložiště). Cílem této sekce je tyto způsoby manipulace s metadatovým uložištěm identifikovat a popsat (není tedy účelem detailní technický popis všech dotazů do metadatového uložiště, ale spíše popis obecných principů a specifických situací - například netradiční zacházení s transakcemi).

\subsubsection{Connector}
\label{sec:ana_connector}
Modul, který je nejblíže metadatovému uložišti, tzv. \textbf{connector} má dvě hlavní zodpovědnosti - zajištění připojení aplikace k uložišti a provádění dotazů nad ním.

%Operations
Modul obsahuje sadu základních dotazů, tzv. \textit{operatinos}, mezi které patří například:
\begin{itemize}
	\item{získání předka uzlu}
	\item{získání atributů uzlu}
	\item{získání sousedních uzlů a hran}
	\item{získání cesty ke kořeni}
	\item{získání podstromu}
\end{itemize}
Tyto operace přímo přistupují do databáze a pomocí programovacího jazyka \textit{Gremlin}\footnote{Používá se \textit{TinkerPop} ve verzi \textit{2.6}.} a jsou na nich postaveny složitější operace nad grafouvou databází. U základních operací nejsou transakce řízeny explicitně, ale implicitně grafovou databází.

% Algorithm
Další částí modulu \textit{connector} jsou tzv. algoritmy, tedy komponenta, pomocí které jsou v metadatovém uložišti hledány samotné datové toky. Tato komponenta řetězí několik grafových algoritmů, přičemž první z nich získá z metadatového uložiště podmnožinu datových toků, která je dalšími algoritmy filtrována a omezována. Tímto způsobem vzniká tzv. \textit{referenční view} - objekt obsahující kompletní graf datových toků pro zadané výchozí uzly a směr datových toků. To je pak využ   íváno dalšími moduly aplikace, například \textit{viewerem} (viz \ref{sec:ana_viewer}), který dle parametrů zadaných ve webové aplikaci graf datových toků vizualizuje.
Jednotlivé algoritmy používají výše popsané základní operace (\textit{operations}) a v některých případech také samy dotazují metadatové uložiště přímo pomocí jazyka \textit{Gremlin}.

% Traversal
Posledním způsobem manipulace s metadatovým uložištěm, který modul umožňuje je přístup pomocí \textit{traverserů}. Ty pracují na obecném principu \textit{traversování} grafů popsaném v kapitole \ref{sec:gdb-dotazy}. V tomto případě ale celý průchod grafem není realizován samotnou grafovou databází, ale přímo aplikací, přičemž grafová databáze je dotazována pouze na dílčí informace - například na okolní uzly. \textit{Traversery} jsou používány v případech, kdy je manipulováno s větší částí grafové databáze, například při jejím exportu. Tyto operace jsou realizovány \textit{vizitory} - každý uzel, který je procházen \textit{traverserem} je následně obsloužen \textit{vizitorem}, který provede požadovanou operaci (jedná se o návrhový vzor \textit{Visitor} - viz \cite{Gamma94}). Obě tyto části, tedy procházení grafu \textit{traverserem} i obsloužení všech objektů \textit{visitorem} jsou prováděny za pomocí základních grafových operací definovaných výše (\textit{operations}). Zároveň je ale také z tohoto kontextu grafová databáze dotazována přímo pomocí jazyka \textit{Gremlin}. Jedná se ale spíše o jednoduché dotazy na dohledání uzlů, jejich atributů apod.

\subsubsection{Merger}
\label{sec:ana_merger}
\textit{Merger} je modul, který je používán při analýze zdrojových systémů. Slouží k zanesení výsledků dílčích analýz jednotlivých částí (např. skriptů) zdrojových systémů do metadatového uložiště.

Vlastní operace \textit{merge}\footnote{Operace \textit{merge} má v kontextu aplikace \textit{Manta Flow} obdobný význam, jako například v \textit{SQL}: pokud objekt není uložen v persistentní vrstvě, je do ní uložen (\textit{insert}), jinak je aktualizován (\textit{update}).} lze zjednodušeně popsat pseudokódem \ref{alg_merger} (správa transakcí a synchronizace je blíže popsána v kapitole \ref{sec:ana_transactions}). Ten je uveden především kvůli složitému transakčnímu modelu, jehož účelem je umožňení provádění dotazů do metadatového uložiště jinými částmi aplikace, zatímco je prováděn \textit{merge} analyzoných částí zdrojových systémů.

Operace se chová různě v případě, kdy je umožňěno verzovaní metadatového uložiště (a to tak obsahuje více revízí) a kdy je vypnuto. V případě zapnutého verzování je \textit{merge} prováděn vždy do nové revize, pokud je verzování vypnuté, je prováděn do hlavní (jediné) revize.

Samotné \textit{merge} operace nad objekty grafové databáze (uzly, hranami, atributy, ...) jsou prováděny přímímy dotazy do databáze pomocí jazyka \textit{Gremlin}. K přístupu do metadadtového uložiště se tedy nevyužívá modul k tomu předurčený - \textit{Connector} (viz \ref{sec:ana_connector}).

\begin{algorithm}
\caption{Merger pseudocode}
\label{alg_merger}
\begin{algorithmic}
	\State $acquireGraphLock()$
	\State $revision\gets getNewestRevision()$
	\If {$revision.isOpen()$}
		\State $acquireScriptsLock()$
		\ForAll{$script in scripts$}
			\State $beginWriteTransaction()$
			\State $merge(script)$
			\State $conditionalCommit()$
		\EndFor
		\State $releaseScriptsLock()$
		\State $beginWriteTransaction()$
		\ForAll{$object in objects$}
			\State $merge(object)$
			\State $periodicalCommit()$
		\EndFor
	\EndIf
	\State $releaseGraphLock()$
\end{algorithmic}
\end{algorithm}

\subsubsection{Viewer}
\label{sec:ana_viewer}
\textit{Viewer} je modul sloužící k poskytování dat uživatelskému rozhraní aplikace (klientské části webové aplikace). Jeho nejčastější interakce s metadatovým uložištěm je dotaz na \textit{referenční view} dle parametrů zadaných uživatelem. To je prováděno pomocí algoritmů definovaných v modulu \textit{connector} (viz \ref{sec:ana_connector}), který obsahuje algoritmy pro hledání datových toků (resp. \textit{referenčního view}).

Kromě toho \textit{viewer} dotazuje metadatové uložiště o další informace, které následně propaguje do uživatelského rozhraní - především o informace o revizích metadatového uložiště a o objekty zdrojových systémů, pomocí kterých uživatel vybírá výchozí uzly pro hledání datových toků (\textit{referenčního view}). Informace o revizích metadatového uložiště jsou dohledávány pomocí základních operací definovaných v modulu \textit{connector} (viz \ref{sec:ana_connector}) a pomocí přímých dotazů do metadatového uložiště pomocí jazyka \textit{Gremlin} s explicitního řízení transakcí. Pro vyhledávání objektů zdrojových systémů (v metadatovém uložišti uzly typu \textit{NODE}, viz \ref{sec:ana_model}) je použito vyhledávání pomocí fulltextového indexu implementovaného pomocí \textit{Apache Lucene}. Ten indexuje uzly v metadatovém uložišti podle jejich názvu a umožňuje jejich rychlé vyhledávání.


\subsubsection{Public API}
\label{sec:ana_public}
\textit{Public API} je modul, který by měl umožňovat vzdálené volání veřejné části funkcionality aplikace pomocí \textit{REST API}. Konkrétně lze tímto způsobem volat například analýzu datových toků mezi různými objekty zdrojového systému. Modul přepoužívá část funkcionality poskytovanou modulem \textit{connector}, část těchto funkcionalit ale duplikuje (s menšími úpravami) a přímo tak dotazuje metadatové uložiště.

\subsubsection{Exporter}
\label{sec:ana_exporter}
Posledním modulem, který přímo interaguje s metadatovým uložištěm je \textit{exporter}, jehož úkolem je exportovat aktuální stav grafové databáze (ne nutně vše, může být exportován například jen interval revizí atd.) buďto do \textit{CSV} souborů, nebo přímo do formátu používaného dalšími nástroji používanými na správu metadat. \textit{Exporter} jako nástroj pro práci s metadatovým uložištěm nejčastěji používá \textit{traversery} a \textit{observery} poskytované modulem \textit{connector}, díky kterým je možné provádět operace nad velkou částí metadatového uložiště bez zásadních paměťových požadavků. Dále jsou využívány základní základní operace (\textit{operations}) poskytované stejným modulem a v některých případech je metadatové uložiště dotazováno přímo pomocí jazyka \textit{Gremlin}.

%TODO dump

%todo možná závěr podsekce s diagramem závislostí modulů

\subsection{Popis ostatních komponent}
\label{sec:ana_other}
\subsubsection{Manta Flow CLI}
Klientská část aplikace \textit{Manta Flow} je \textit{command line} aplikace implementovaná v programovacím jazyce Java sestavená z několika modulů. Jejím hlavním úkolem je extrakce skriptů ze zdrojových databázových systémů a repozitářů (modul \textit{Exktractor}), a jejich následné parsování a analýza (modul \textit{Analyzer}). Analýza skriptů probíhá v klientské části aplikace z toho důvodu, že využívá vyextrahované slovníky objektů zdrojového systému, které by v případě provádění analýzy serverovou částí aplikace musely být přenášena na server spolu se skripty. Poté, co proběhne zpracování (\textit{merge} - viz. kapitola \ref{sec:ana_merger}) zanalyzovaných skriptů serverem, výsledky jsou vyexportovány zpět do klientské části aplikace, odkud mohou být případně nahrány do externí metadatové databáze.

\subsubsection{Configurator}
\textit{Configurator} je \textit{standalone} webová aplikace implementovaná v programovacím jazyce Java (jako třívrstvá aplikace), jejímž úkolem je poskytnout \textit{GUI (grafické uživatelské rozhraní)}, pomocí kterého může uživatel změnit komplexní konfiguraci aplikace \textit{Manta Flow}. Konfigurace je typicky obsažena v \textit{properties} souborech a to na různých místech. Serverová a klientská část \textit{Manta Flow} má vlastní konfiguraci. \cite{Molitor18}

\subsubsection{Updater}
\textit{Updater} je \textit{standalone} webová aplikace implementovaná v programovacím jazyce Java (jako třívrstvá aplikace), jejímž úkolem je poskytnout \textit{GUI}, které provede uživatelem \textit{updatem} aplikace \textit{Manta Flow} (konkrétně její serverové části) na novější verzi. Aplikace umožňuje uživateli provést změny v komplexní konfiguraci aplikace a provede \textit{merge} těchto změn s původním nastavením. \cite{Gondek16}


%TODO
\subsection{Požadavky a omezení}
TODO

\subsubsection{Datová konzistence}
\label{sec:ana_transactions}
Z popisu jednotlivých komponent aplikace v kapitolách \ref{sec:ana_components} a \ref{sec:ana_other} je patrné, že je používáno několik heterogenních přístupů k manipulaci s daty (především v grafové databázi) a je tak třeba pečlivě řídit konzistenci dat. Primárním princepem zaručujícím konzistenci dat je verzování modelů datových toků na jednotlivé \textit{major} a \textit{minor} revize \cite{Sykora17}.
Cílem je, aby všechny moduly, které pouze provádějí analýzu datových toků a tu předávají (v některé z dostupných forem) dále, tedy \textit{Viewer}, \textit{Exporter} a \textit{Public API}\footnote{Pomocí \textit{Public API} může být grafová databáze i upravována, všechny úpravy jsou ale prováděny pouze pomocí modulu \textit{Merger}.} mohly být spouštěny nad některou z uzavřených (\textit{commitnutých}) revizí, zatímco \textit{Merger}, který vkládá nové informace na základě analýzy provedené klientskou částí aplikace, má k dispozici jednu otevřenou \textit{necommitnutou} pracovní revizi. Všechny zmíněné moduly používají pro přístup do grafové databáze primárně modul \textit{Connector}, který obsahuje operace čtení i zápisu, obecně ale platí, že operace zápisu jsou používány pouze \textit{Mergerem}.
I přes systém revizí může vzniknout často konkurence při čtení a zápisu dat do grafové databáze:

\begin{itemize}
	\item{\textit{Úprava uzavřené revize}}: Při vytváření nové revize, která vzniká typicky jako \textit{full update} celého modelu datových toků\footnote{Nová \textit{major} revize vzniká jako \textit{full update} modelu datavých toků, nová \textit{minor} revize jako \textit{incremental update}. Konkrétní implementace obou operací se v některých detailech liší, z pohledu datové konzistence ale řeší oba přístupy koncepčně stejný problém. Můžeme proto analýzu vzniku nové revize datového modelu zobecnit na \textit{full update} přístup - tedy na \textit{major} revize.}, může nastat několik situací. Připomeňme, že platnost vztahu mezi dvěmi uzly je parametrem hrany spojující tyto uzly. Pokud se libovolné dva uzly a jejich vztah nezmění mezi uzavřenou revizí \textit{A} a vznikající neuzavřenou revizí\textit{B}, pak jsou upraveny atributy hrany reprezentující tento vztah a může tak vzniknout konkurence při čtení revize \textit{A} a zápisu do revize \textit{B}.
	\item{\textit{Kombinace operací pro čtení a zápis při vytváření nové revize}}: Modul \textit{Merger} při vytváření nové revize modelu datových toků v grafové databází spouští několik komplexních postproccessingových algoritmů, které vyžadují přístup pro čtení i zápis do pracovní otevřené revize (a jsou v některých případech prováděny paralelně).
	\item{\textit{"Stop the world" operace}}: Je také definováno několik takzvaně "stop the world" operací, tedy operací, při kterých je znepřístupněna velká část, nebo přímo celá grafová databáze. Mezi tyto operace patří odstraňování starých revizí modelu datových toků, import či export kompletního \textit{dumpu} grafové databáze.
\end{itemize}

Vzhledem k výše uvedeným situacím je konkurence při přístupu k datům řízena dvěmi dalšími mechanikami, kterými jsou databázové transakce a synchronizace pomocí \textit{reentrant} zámků.

Transakce jsou řízeny programaticky\footnote{Kromě programatického řízení transakcí je možné použít i deklarativní řízení transakcí (\textit{Declarative Transaction Model}). Ne pro všechny grafové databáze ale v tuto chvíli existují implementace, které by deklarativní transakce v rámci \textit{Spring} frameworku umožňovaly. V tuto chvíli je podporuje například \textit{Orient DB} (\url{https://github.com/orientechnologies/spring-data-orientdb}) a databáze přístupné pomocí \textit{TinkerPop Gremlin 2.x} (\url{https://github.com/gjrwebber/spring-data-gremlin}). Podpora pro \textit{Apache TinkerPop 3.x} je aktuálně ve vývoji.} (\textit{Programatic Transaction Model} \cite{Little04}) za pomoci implementace \textit{Spring TransactionTemplate}\footnote{\url{https://docs.spring.io/spring/docs/current/javadoc-api/org/springframework/transaction/support/TransactionTemplate.html}} kombinující \textit{Titan} transakce a Java \textit{reentrant} zámky.
\textit{Titan} v případě transakcí pouze přeposkytuje funkcionalitu používané podkladové databáze, kterou je v případě \textit{Manta Flow} embedded NoSQL databáze typu klíč-hodnota \textit{Persistit}\footnote{\url{https://github.com/pbeaman/persistit}}. Ta, stejně jako mnoho dalších grafových databází, implementuje \textit{Multi-Version Concurrency Control (MVCC)} \cite{Prakash10} využívající \textit{Snapshot Isolation}\footnote{Všechny záznamy v databázi jsou verzovány. Při čtení transakce vytvoří kopii poslední \textit{commitnuté} verze čteného záznamu, ta je po ukončení transakce odstraněna. \textit{Snapshot isolation} tak v zásadě odpovídá úrovni izolace transakcí \textit{Read committed}.} úroveň izolace transakcí.
Jedná se o \textit{optimistický} transakční model, žádné databázové objekty (ani záznamy) nejsou zamykány, takže transakce probíhají v plné rychlosti bez čekání. Může tak ale nastat situace, že ne všechny transakce je možné \textit{commitnout} a jedna (nebo více) z těchto transakcí tak musí být zrušena (a zopakována). \textit{Snapshot} izolace transakcí má v grafových databázích také další specifický důsled. Zatímco v případě relačních databází vždy existuje unikátní identifikátor každého záznamu (byť i implicitní), v grafových databázích tomu tak není - tedy neexistuje žádná unikátnost uzlů a hran v grafové databázi za předpokladu, že není explicitně vynucena pomocí unikátních indexů. Ty nejsou v aplikaci \textit{Manta Flow} používány, může tak nastat situace, kdy se dvě transakce souběžně snaží o vytvoření (z pohledu aplikační logiky) identického uzlu či hrany, obě transakce uspějí (nedojde k jejich kolizi) a objekt je tak vytvořen dvakrát (čímž je zanesena nekonzistence do metadatového uložiště).
Aby k těmto konfliktům paralelních transakcí nedocházelo, používá aplikace při přidělování transakcí synchronizaci pomocí \textit{reentrant} zámků, přičemž vlastní zámek je pro obecné transakce a vlastní pro \textit{read-only} transakce. Nemohou tak současně existovat dvě transakce, které by mohly do databáze zapisovat zároveň - což může potenciálně velmi snižovat rychlost prováděných operací. Důsledkem tohoto systému zámků je serializace paralelních klientských požadavků pro zápis do metadatového uložiště (\textit{merge}). Tento process v klientské části aplikace přitom probíhá paralelně, tipicky minimálně ve čtyřech vláknech.
% TODO poznámka k performance Zároveň se jedná o proces, u kterého je rychlost jedním ze zásadních kritérií - jedná se o noční proces, v některých případech přitom trvá více než pět hodin.

Další používanou synchronizační mechanikou je zamykání grafové databáze (respektive uzlů reprezentující jednotlivé revize) \textit{reentrant read-write} zámkem při několika operacích, které jsou součástí \textit{Mergeru} a importu a exportu \textit{dumpu} grafové databáze. Tyto zámky tak společně s verzováním modelu metadat slouží při \textit{mergi} de-facto jako implementace \textit{long living} transakcí - samotný \textit{merge} objektů do grafové databáze je prováděn v několika transakcích\footnote{Transakce jsou objektem uloženým v operační paměti, přičemž velikost transakce narůstá s počtem úprav v databázi.}.

Poslední synchronizace na straně serveru je zamykání metadatové databáze nesoucí informace a zanalyzovaných skriptech pro účely kontroly licencí. K této synchronizaci dochází při \textit{mergi} v okamžiku, kdy jsou informace o skriptech do metadatového uložiště nahrávány.


\subsubsection{Výkon aplikace}
\label{sec:ana_performance}
%TODO


\subsubsection{Škálovatelnost}
\label{sec:ana_scaling}
%TODO

\subsubsection{Modifikovatelnost}
\label{sec:ana_modularity}
%TODO


\subsubsection{API grafových databází}
\label{sec:ana_gdbapi}
\textit{Manta Flow} pro dotazování grafové databáze (v tuto chvíli \textit{Titan}) používá programovací jazyk \textit{Gremlin} ve verzi \textit{2.6}. Lze tedy říci, že tento programovací jazyk slouží jako API mezi aplikací a aktuálně používanou grafovou databází. Z benchmarků dalších grafových databází \cite{Kovar18} vyplývá, že bude-li současná grafová databáze nahrazena, jejím nástupcem bude pravděpodobně \textit{JanusGraph}, nebo \textit{OrientDB}\footnote{Obě zmíněné databáze jsou blíže popsány v kapitole \ref{sec:gdb-databaze}.}. Obě tyto databáze podporují programovací jazyk \textit{Gremlin}, \textit{JanusGraph} ve verzi \textit{3.x}, zatímco \textit{OrientDB} ve verzi \textit{2.x}.
Protože \textit{Gremlin} a API, která mají grafovým databázím zaručit jeho podporu, prošli mezi verzemi \textit{2.x} a \textit{3.x} zásadními změnami (viz \ref{sec:gdb-jazyky}), budou v této sekci popsány obě tyto technologie. Obě verze jazyka \textit{Gremlin} poskytují \textit{Java API} (\textit{Gremlin 2.x \cite{Gremlin14}, Gremlin 3.x \cite{Gremlin17}}). Pro přístup do datové databáze budou tedy v práci nadále používána tato dvě API.

% TODO procesní diagramy

%security
% - podle rolí
% - nevidím flow v cizích systémech, ale vidím, že z vlastního tam flow vede

% - Není nutné upravovat algoritmus pro hledání flow  				--- stačí najít kompletní flow se všmi objekty a odfiltrovat zakázané.
% - Úprava algoritmu pro hledání flow by alg. nezrychlila 			--- stejně by bylo nutné hledat přes zakázané objekty (pro případ, že za nimi jsou ještě viditelné objekty, které jsou součástí flow)
% ==> Neupravovat algoritmy, natož DB dotazy
% ==> Nalezená flow by měla být filtrována -> sw. vrstva - na jaké úrovni?
% - Filtrovat by se mělo až ve chvíli, kdy je flow kompletní
% --



%\subsection{Architektonická omezení}
% Podněty
% - Architektura vyhovující cloudovým požadavkům
%  - žádný filesystem, jen pipes na jiné služby, které mohou filesystem podporovat (lze využívat pouze temp uložiště a sfree buckety)
%  - např jedna služba client, jedna server, jedna grafovka
%  - jak by se řešily veškeré konfigurace

% Škálovatelnost - cloud
% https://acloud.guru/
% https://www.amazon.com/Patterns-Enterprise-Application-Architecture-Martin/dp/0321127420
% https://martinfowler.com/articles/microservices.html
% https://martinfowler.com/tags/application%20architecture.html - Serverless Architectures on AWS
% https://www.youtube.com/watch?v=LAWjdZYrUgI
% AWS Lambda - amazon SaaS (?)
% S3 - cloud storage (?)
% Cloud computing http://nvlpubs.nist.gov/nistpubs/Legacy/SP/nistspecialpublication800-145.pdf
% JPMC GAIA https://www.americanbanker.com/news/unexpected-champion-of-public-clouds-jpmorgan-cio-dana-deasy
% Cloud Awareness ;  OpenStack

%%%%%%%%%%%%%%%%%%%%%%%%%%%%%%%%%%%%%%%%%%%%%%%%%
\chapter{Návrh architektury}
Cílem této kapitoly práce je navrhnout a popsat architekturu, která bude řešit, či umožní budoucí řešení problémů popsaných v kapitole \ref{sec:ana_problems} a vyhoví tak požadavkům na architekturu aplikace definovaným v kapitole \ref{sec:ana_requiremets}.

Tyto požadavky jsou rozděleny na tři skupiny, přičemž řešení požadavků mezi jednotlivými skupinami na jsou na sobě bůďto navzájem nezávislá, nebo je případná závislost explicitně uvedena v následujícím textu.

První jsou problémy způsobené mícháním perzistentní a byznys logiky v rámci serverové části aplikace, mezi které patří především špatná \textit{modifikovatelnost} aplikace a nízká \textit{viditelnost} interakcí jednotlivých komponent serverové částí aplikace. Také jsou diskutovány dopady těchto omezení na \textit{výkon} aplikace. Konkrétně do této skupiny patří požadavky \textit{N1, N2, N3, N5, N6} a \textit{N8}. Tyto požadavky jsou realizovány především změnou architektury komponenty \textit{Connector} (sekce \ref{sec:ana_connector}) popsanou v sekci \ref{sec:des_api} této kapitoly.

Do druhé skupiny patří omezení vyplývající z aktuální architektury \textit{Manta Flow} na úrovni orchestrace jednotlivých aplikací, které jsou součástí celého řešení. Nově vzniklé podpůrné aplikace \textit{Configurator} (kapitola \ref{sec:ana_configurator}) a \textit{Updater} (kapitola \ref{sec:ana_updater}) zatím nejsou ukotveny v architektuře celého řešení.
Do této skupiny patří požadavky \textit{F1, F2, F3} a \textit{N7}. Návrh řešení této skupiny požadavků je popsán sekcí \ref{sec:des_orchestration} této kapitoly.

Třetí část této kapitoly je věnována požadavku \textit{N4} a navrhuje možnosti horizontálního škálování aplikace - sekce \ref{sec:des_scaling}.

\section{Úprava architektury komponenty Connector}
\label{sec:des_api}
Jak bylo uvedeno v sekci \ref{sec:ana_problems}, jedním z klíčových problémů, kterým v současné době aplikace \textit{Manta Flow} čelí, je splývání byznys logiky aplikace s perzistenční logikou, která implementuje ukládání datových toků do grafové databáze a jejich dotazování. To je nejvíce patrné v modulu \textit{Connector}, který zajišťuje připojení aplikace k grafové databázi, vkládání a dotazování dat do/z grafové databáze (perzistenční logika) a poskytuje způsoby analýzy dat uložených v grafové databázi (algoritmy a traversaly - \ref{sec:ana_connector}), obsahuje tedy také značné množství byznys logiky serverové části aplikace. První částí návrhu architektury je tak změna architektury \textit{Manta Flow Serveru} tak, aby byl tento modul nahrazen vícevrstvou architekturou, která:

\begin{itemize}
   \item umožní připojení aplikace do grafové databáze,
   \item definuje operace, které tvoří persistenční logiku aplikace pomocí \textit{API} a zamezí přímému přístupu do grafové databáze z jiných komponent,
   \item umožní přidělení/zamezení přístupu ke konkrétním informacím obsaženým v grafové databázi na základě definovaných oprávnění uživatelů,
   \item poskytne funkcionalitu pro pokrytí všech požadavků na přístup k datům ostatních modulů serverové části aplikace.
\end{itemize}

Hlavní kvalitativní kritéria zvolené architektury, která vycházejí z těchto požadavků, jsou:

\begin{itemize}
   \item{\textit{Jednoduchost}}: Zvolená architektura musí maximalizovat princip separace zájmů. Tím bude výrazně snížena závislost (\textit{coupling}) byznys logiky aplikace a persistenční logiky a bude docíleno jednoduchosti komponent.
   \item{\textit{Viditelnost}}: Komunikace mezi jednotlivými komponentami musí být přímočará a musí být rozšířitelná o novou komponentu. Aktuálně existuje požadavek na začlenění komponenty, která bude řídit přístup uživatele k informacím uloženým v metadatovém úložišti na základě oprávnění uživatele.
   \item{\textit{Modifikovatelnost}}: Jednotlivé komponenty musí být jednoduše rozšířitelné. Musí být snadné přidávání komponent. Implementace (některých) komponent musí být zaměnitelná bez dopadů na další komponenty (jedná se především o implementaci komponenty implementující perzistenční logiku aplikace - kvůli možné výměně grafové databáze a tedy i dotazovacího jazyka).
   \item{\textit{Výkon}}: V kapitole \ref{sec:ana_performance} jsou popsány požadavky na výkon (především) serverové části aplikace \textit{Manta Flow}. Architektura musí umožňovat splnění těchto požadavků a musí být definovány možné způsoby optimalizace výkonu.
\end{itemize}

\subsection{Transakční model a řízení konzisence dat}
\label{sec:des_transactions}
Jak je popsáno v kapitole \ref{sec:ana_transactions}, aplikace \textit{Manta Flow} má velmi specické požadavky týkající se řízení datové konzistence metadatového úložiště. Jedná se o problém, který je koncepční a může mít podstatné dopady na architekturu aplikace.

Bylo uvedeno, že stávající řešení používá \textit{PTM} transakční model, transakce jsou mezi jednotlivými komponentami serverové části aplikace propagovány jako objekty a jsou používány k přímému přístupu do databáze. Cílený stav je takový, aby byl přímý přístup do databáze možný pouze z komponenty k tomu určené a ostatní komponenty (obsahující byznys logiku aplikace) pro přístup do databáze vždy používaly tuto komponentu. Současně ale musí být umožněno propagování transakcí mimo tuto komponentu, transakce jsou často rozsáhlé\footnote{Úroveň izolací \textit{snapshot isolation} vyplývající z \textit{MVCC} implementace transakcí používané grafovými databázemi je specifická v tom, že je v průběhu transakce duplikováno velké množství databázových objektů. Ty jsou navíc často ukládány do paměti klientské služby, nikoliv databáze. Pro optimalizaci přístupů do grafové databáze je tak podstatné zvolení správné velikosti transakcí. V případě minimalistických transakcí zahrnujících jednotky operací je režie transakcí příliš velká a práce s databází není efektivní. Při příliš rozsáhlých transakcích dochází k vyčerpání operační paměti klientské služby kvůli duplikování databázových objektů.}. Je tedy nutné zavést mechanismus \textbf{abstrakce transakcí}.
V aplikacích používajících relační databáze je k tomuto účelu standardně používán \textbf{deklarativní transakční model}. Ten umožňuje konfigurovat chování každé metody, která přistupuje do datového zdroje (v tomto případě databáze), nebo která takovou metodu volá. U každé takové metody je definováno, jak má být transakce propagována, jaký je stupeň izolace transakce, zda je \textit{read-only} a v jakém případě dochází k \textit{rollbacku} transakce. Implementace tohoto modelu frameworkem \textit{Spring} je popsána v dokumentaci \cite{SpringTransactions}.
V kontextu grafových databázích ale pro tento model zatím není u řady databází podpora a o standardní řešení se nejedná. Součástí návrhové fáze tak bylo vytvoření dvou \textit{PoC} implementací (viz příloha \ref{apx:cd}), které ověřují použitelnost tohoto řešení pro dotazovací jazyk \textit{Gremlin} ve verzi \textit{2.x} a \textit{3.x}. V obou případech bylo ověřeno, že použití \textit{deklarativního transakčního modelu} v kombinaci s frameworkem \textit{Spring} je pro aplikaci \textit{Manta Flow} vhodným řešením. Podpora pro jazyk \textit{Gremlin} ve zmíněných verzích, respektive pro databáze, které ho podporují, byla v rámci těchto \textit{PoC} implementována vlastní - existující implementace (uvedené v kapitole \ref{sec:ana_transactions}) jsou dostupné pouze v experimentálních verzích a obsahují chyby.
Důležitým omezením, které vyplývá ze zvoleného řešení je, že všechny komponenty, do kterých jsou propagovány transakce, musí být součástí monolitické architektury - není možná propagace transakcí pomocí standardních komunikačních protokolů (například \textit{HTTP/s}). Navržená pravidla pro práci s deklarativními trasakcemi jsou:

\begin{itemize}
   \item Všechny metody přistupující do databáze musí mít nakonfigurované transakční chování.
   \item Všechny metody zapisující do databáze musí mít nastaven způsob propagace na \textit{Mandatory} - všechny metody,
   které tyto využívají tak musí mít také nakonfigurované transakční chování.
   \item Všechny metody, které používají transakční metody se způsobem propagace \textit{Mandatory} musí mít konfigoravaný
   stejný způsob propagace, pokud nejsou součástí komponenty implementující byznys logiku aplikace. Tím je zaručeno, že rozsah
   transakcí provádějících změny v databázi je řízen právě v komponentách realizujících byznys logiku aplikace (zpravidla se bude jednat o komponentu \textit{Merger}).
\end{itemize}


Dalším faktorem ovlivňující architekturu serverové části aplikace, který se týká datové konzistence, je systém explicitních zámků. Ten zajišťuje synchronizaci (serializaci) paralelních přístupů do grafové databáze - na úrovni celé databáze může v jednu chvíli existovat pouze jedna zapisovací transakce (popsáno v kapitole \ref{sec:ana_transactions}). Tento systém je velkým omezením pro architekturu aplikace, protože zabraňuje jejímu (efektivnímu) škálování. Systém zámků je nutný - pokud by nebyl používán, docházelo by k zanášení nekonzistencí do metadatového úložiště a to především k duplikacím objektů\footnote{Potenciální alternativou k systémů zámků by bylo zavedení unikátních identifikátorů uzlů a vytvoření indexů hlídajících tuto vlastnost. V tom případě by nedocházelo k duplikaci uzlů a používání zámků by nebylo nutné. Používání unikátních indexů v grafových databázích ale není obecně příliš efektivní. Uzly typu \textit{NODE} navíc žádné přirozené unikátní identifikátory nemají, respektive jejich vytvoření by vedlo často na řetězce obsahující stovky znaků. Tento přístup byl tedy zavržen.}. Je tedy nutné tento systém upravit tak, aby lépe vyhovoval požadavků aplikace. Konkrétně byl navržen algoritmus pro zamykání objektů uložených v metadatovém úložišti definovaný následujícím chováním:

\begin{itemize}
   \item zamykány jsou uzly v grafové databázi,
   \item existují dvě úrovně zámků - pro čtení a pro zápis,
   \item zámky pro čtení jsou uplatňovány pouze v případě, že je čteno z \textit{necommitnuté} revize,
   \item při úpravě vlastností uzlu (zpravidla úprava intervalu platnosti uzlu) je zamykán tento uzel,
   \item při vytváření novéhu uzlu je zamykán uzel, který je předkem nového uzlu,
   \item při mazání uzlu je zamykán uzel, který je předkem mazaného uzlu,
   \item při přidávání hran, které nejsou součástí hiearchie grafu datových toků (hrany typu \textit{DIRECT}, \textit{FILTER} a \textit{MAPS\_TO}), je zamykán výchozí uzel nové hrany,
   \item v případě situace vedoucí k potencionálnímu \textit{dead-locku} dojde ke \textit{commitu} všech zůčastněných transakcí a zpracování dalších uzlů v nové transakci
\end{itemize}

Operace, které by i nadále měly zamykat celou databázi jsou promazávání starých revizí a export či import kompletního dumpu databáze.

Takto navržený systém zámků umožní paralelní zápis do metadatového úložiště (byť zrychlení kvůli zamykání objektů a tedy potenciálnímu čekání v praxi nebude dosahovat počtu paralelních procesů). Existuje ale nadále silná závislost mezi architektonickými omezeními aplikace a omezeními na implementaci zámků. \textit{Gremlin} (v žádné verzi) neposkytuje vlastní řešení zámků v grafových databázích. Zamykat objekty databáze pomocí databázových zámků je tak možné pouze u některých grafových databází. Nabízí se tak možnost sychnronizace pomocí \textit{Java} konstruktů, jakou jsou \textit{synchronized} metody/bloky, zámky, atomické proměnné atd. Tyto nástroje je ale možné použít pouze v případě zachování monolitické architektury serverové části aplikace. V případě jejího rozdělení na  komponety spouštěné na různých \textit{JVM} by bylo pro synchronizaci zámků nutné používat externí nástroje jako jsou relační databáze, nebo specifické nástroje jako \textit{Redisson}\footnote{\url{https://redisson.org/}}.

Vzhledem k tomu, že systém zámků řeší konzistenci dat z pohledu byznys logiky aplikace (integrita dat je zajištěna používáním transakcí), měl by být implementován komponentou obsahující tuto byznys logiku - konkrétně modulem \textit{Merger} (kapitola \ref{sec:ana_merger}), který může jako jediná komponenta implementující byznys logiku aplikace pracovat s \textit{necommitnutou} revizí metadatového úložiště.

\subsection{Návrh komponent}
\label{sec:des_components}
V zádání diplomové práce je navržno, aby byl jako architektonický styl pro návrh architektury splňující uvedená kritéria zvolena vícevrstvá architektura. Ta uvedeným kritériím zcela vyhovuje. Navržená architektura je popsána \textit{UML\footnote{Unified Modeling Languege. Používaná notace 2.5.1 \cite{UML17}} diagramem komponent} \ref{fig:des-connector} a jednotlivé její komponenty jsou popsány v následujícíh podkapitolách.

\begin{figure}
\begin{center}
\includegraphics[width=10cm]{figures/connector_modules}
\caption{Upravená architektura modulu Connector}
\label{fig:des-connector}
\end{center}
\end{figure}

\subsubsection{Doménový model}
\label{sec:des_domain}
Jedním z důsledků splývání byznys logiky a persistenční logiky je absence doménového objektového modelu serverové části aplikace. Místo něj je používán obecný model založený na třídách \textit{Vertex} (vrchol) a \textit{Edge} (hrana). Tato reprezentace dat v aplikaci má dva zásadní dopady:

\begin{itemize}
   \item Model je příliš obecný (dokáže pojmout graf jakékoliv struktury, potažmo graf bez definované struktury), datový model metadatového úložiště je ale přesně specifikovaný (diagram \ref{fig:ana-model}) a jeho nedodržení znamená zanesení nekonzistence a tedy riziko selhání aplikace.
   \item Vzhledem k tomu, že zmíněný model poskytuje jako \textit{SPI} dotazovací jazyk \textit{Gremlin} (respektive \textit{API Blueprints}) a to je implementováno \textit{Java} knihovnou pro práci s grafovou databází \textit{Titan}, umožňují instance tohoto modelu přímý přístup do grafové databáze a umožňuje tedy obcházení metod dedikovaných pro práci s grafovou databází.
\end{itemize}

Je evidentní, že tento doménový model není pro navrženou architekturu vhodný, byl tak vytvořen vlastní, \textbf{specifický doménový model}.

Základními omezeními návrhu doménového modelu jsou:
\begin{itemize}
   \item{\textit{Rich model}}: Doménový model je navržen jako tzv. \textit{rich domain model}, obsahuje nejen definice entint a jejich parametrů, ale také základní logiku, kterou tyto entity a vazby mezi nimi představují.
   \item{\textit{POJO model}}: Doménový model by měl být tvořen pouze \textit{POJO třídami} \footnote{\textit{POJO (Plain Old Java Object)} je označení pro \textit{obyčejný Java objekt}, tedy objekt, který není \textit{J2EE Bean, Spring Bean, Entity Bean,}  atd.}.
   \item{\textit{Model neobsahuje perzistenční logiku}}: Součástí doménového modelu by neměla být perzistenční logika. Toto omezení je v podstatě důsledkem omezení na \textit{POJO} objekty - obsahoval-li by doménový model perzistenční logiku, musel by nutně obsahovat také transakční logiku, která je ale navržna tak (kapitola \ref{sec:des_transactions}), že není realizovatelná pomocí \textit{POJO} objektů. Perzistenční logika je tak realizovaná samostatnou komponentou (kapitola \ref{sec:des_persistence}).
\end{itemize}

\subsubsection{Databázová vrstva}
\label{sec:des_database}
Databázová vrstva slouží pouze pro připojení aplikace do databáze a pro dodání podpory pro externí indexovací nástroje, pokud je potřeba. Základní rozhraní této vrstvy tvoří \textit{TransactionManager}, díky kterému jsou podporovány \textit{deklarativní transakce} ve vyšších vrstvách aplikace a rozhraní reprezentující grafovou databázi\footnote{V případě jazyka \textit{Gremlin 2.x} poskytuje přístup do databáze rozhraní \textit{com.tinkerpop.blueprints.Graph}.} jako vstupní bod pro dotazy do databáze.

Tato funkcionalita je vyčleněna do samostatné vrstvy především kvůli principu separace zájmů. Také v některých případech umožňuje výměnu grafové databáze bez úprav vyšších vrstev - jedná se o případy, kdy je možné obě databáze dotazovat pomocí stejné verze jazyka \textit{Gremlin}.

\subsubsection{Perzistentní vrstva}
\label{sec:des_persistence}
Jak je zmíněno v kapitole \ref{sec:ana_state_of_art}, žádný ze zkoumaných nástrojů pro abstrakci objektově-grafového mapování nevyhovuje požadavkům \textit{Manta Flow} a navržená architektura tak žádný takový nástroj nepoužívá. Místo toho je navržena vlastní softwarová vrstva, která toto mapování provádí. Tou je právě \textit{Perzistentní vrstva}. Jejím úkolem je poskytnout \textit{API}, které pokryje všechny požadavky na dotazy do grafové databáze (a jeho implementaci). \textit{Perzistentní vrstva} je \textit{uzamčená vrstva} (viz \ref{sec:n-tier}), nemůže tedy být obcházena vyššími vrstvami při přístupu k nižší vrstvě.

\textbf{API je tvořeno sadou \textit{repository} objektů}, které implementují (mimo jiné) \textit{CRUD} operace a jsou inspirovány návrhovým vzorem \textit{Repository pattern} \cite{Dorfmann16}. Pro každou entitu reprezentovanou doménovým modelem existuje jeden \textit{repository} objekt, přičemž každý z těchto objektů obsahuje (pokud je to pro příslušnou entitu relevantní) následující typy metod:

\begin{itemize}
   \item{\textbf{Create}}: Metody slouží pro vytváření nových instancí entit (uzlů a hran) v grafové databázi. Parametrem je samotná instance obsahující parametry uzlu/hrany, interval platnosti uzlu/hrany a pokud existují, tak uzly, které jsou přímímy předky v hiearchii stromu grafových toků (v takovém případě je mezi těmito uzly vytvořena hrana). Například při vytváření nové instance entity \textit{Node} je uzel spojen s rodičovckou instancí entity \textit{Node} a/nebo s instancí entity \textit{Resource}\footnote{Již dříve bylo uvedeno, že graf datových toků ve skutečnosti nesplňuje stromovou strukturu. Je tak například možné, že entita \textit{Node A} bude mít rodičovskou entitu \textit{Node B} a \textit{Resource C}, přičemž \textit{B} a \textit{C} nejsou součástí stejného podstromu grafu datových toků.}.

   \item{\textbf{Find}}: Vstupem \textit{find} metod je jeden nebo více parametrů specifických pro danou entitu, přičemž kombinací těchto parametrů musí být vždy unikátně identifikovatelná maximálně jedna instance entity. Příkladem může být (jazykem \textit{Gremlin} generované) id instance (jakékoliv entity), název entity \textit{Resource}, nebo plně kvalifikované jméno entity \textit{Node}. Metoda pak vrací nalezenou instanci entity - pokud existuje.

   \item{\textbf{Update}}: Operace \textit{update} je u většiny entit redukována na úpravu intervalu platnosti entity, respektive na úpravu konce tohoto intervalu (začátek intervalu je po vytvoření instance neměnný). Ten je upravován při \textit{updatu} (ať už úplném, nebo inkrementálním) metadatového úložiště operací \textit{merge}. V případě, že se jedná o inkrementální \textit{update}, je možné operaci volat s parametrem definujícím, že operace bude rekurzivní - bude uplatněna na celý podstrom entity (včetně entity samotné). Pokud by úpravou konce intervalu platnosti došlo k situaci, že by instance entity nebyla platná již v žádné revizi, je instance smazána pomocí operace \textit{delete}.

    U entity \textit{Flow} je navíc možné přidávat uživatelsky definované parametry entity. Tyto parametry nejsou součástí doménového ani datového modelu aplikace a nejsou ani využívány žádnými algoritmy byznys logiky aplikace. Všechny ostatní parametry (všech entit) jsou jasně definované v doménovém i datovém modelu aplikace a jsou nastavovány při vytváření instance entity operací \textit{create} a dále jsou neměnné.

   \item{\textbf{Delete}}: Metody typu \textit{delete} odstraňují entity z grafové databáze bez ohledu na jejich interval platnosti. Operace může být (stejně jako operace \textit{update}) rekurzivní.

   \item{\textbf{Get*}}: Metody typu \textit{get*} slouží k dohledání entit, které jsou s vstupní entitou propojeny. Příkladem může být dohledání potomků (či předků) entit \textit{Node} a \textit{Resource}, dohledání parametrů (entit \textit{Attribute}) entity \textit{Node}, nebo dohledání entit, které jsou s vstupní entitou propojeny hranami \textit{Flow}, nebo \textit{MapsTo}.

   Metody typu \textit{get*} mají zpravidla další argumenty, které slouží k filtrování dohledávaných entit. Pokud kombinace těchto argumentů vytváří unikátní identifikaci instancí entity v rámci kontextu dotazu (vstupní entity), tak je podle jmenné konvence součástí názvů metody jednotné číslo dohledávané entity (například metoda \textit{getChild}). Pokud argumenty unikátní identifikátor netvoří, je součástí názvu metody množné číslo dohledávané entity (například metoda \textit{getChildren}).

   \item{\textbf{Index Search}}: Zatímco ostatní metody pro dotazování dat z grafové databáze (\textit{find, get*, query}) implicitně využívají existujících interních indexů grafové databáze (a perzistentní vrstva od nich tak odstiňuje vyšší vrstvy), pro využítí externích indexů (v případě \textit{Manta Flow} se jedná o \textit{Apache Lucene}) je nutné definovat vlastní metody. Argumenty těchto metod jsou definovány přesně dle definic těchto externích indexů tak, aby byl plně využit jejich potenciál. Případné další filtrování výsledků těchto dotazů (na základě parametrů neobsažených v externích indexech) tak musí probíhat již v komponentách realizujících byznys logiku aplikace.

   \item{\textbf{Query}}: Metody typu \textit{query} slouží k obecnému dotazování entit. Argumenty těchto metod mohou být filtry  na libovolné parametry entit, včetně parametrů entit s nimi spojenými (hranami a uzly). Na základě těchto argumentů je vygenerovaný databázový dotaz, jehož součástí je uplatnění všech těchto filtrů - jedná se tak o výrazně efektivnější přístup než je například dotázání všech uzlů z grafové databáze a jejich následné filtrování ve vyšších vrstvách aplikace. Jedná se o nejobecnější nástroj, který API pro dotazování dat z grafové databáze nabízí.
\end{itemize}

% struktura
Samotná perzistentní vrstva je dále rozčleněna na komponenty (viz diagram \ref{fig:des-persistence}). Účelem tohoto rozdělení je striktní oddělení \textit{API} komponenty a jeho implementace. Aplikace \textit{Manta Flow} používá pro správu závislostí nástroj \textit{Maven}\footnote{\url{https://maven.apache.org/}}, u jednotlivých modulů je tak definovány, jakým způsobem by měly být referencovány. Analogicky s tímto návrhem by měly být navržena také další vrstvy, které mohou být v budoucnosti do architektury přidávány. Jsou definovány tři moduly:

\begin{itemize}
   \item{\textit{API}}: Modul definuje rozhraní pro používání perzistentní vrstvy. Vyšší vrstvy využívající perzistenční vrstvu by měly referencovat právě tento modul.
   \item{\textit{Test}}: Modul obsahuje třídy s definicemi testů pokrývajících funkcionalitou \textit{API} - bez využití jakékoliv implementace \textit{API}. Jednotlivé implementace \textit{API} potom tyto testy implementují (poskytují implementaci \textit{API}), čímž je zajištěno, že je vždy testováno chování \textit{API}, a ne jen specifické detaily poskytnutých implementací. Modul závisí na modulu \textit{API}.
   \item{\textit{Implementace}}: Modul obsahující implementaci vrstvy. Vyšší vrstvy využívající perzistenční vrstvu by tento modul měly referencovat, pouze ale s parametrem \textit{scope=provided}. To zaručí, že není možné v kódu komponent používajích perzistenční vrstvu používat třídy, které jsou součástí této třídy a je tak zaručeno, že definované \textit{API} není obcházeno. Modul je závislý na modulu \textit{API}, modulu \textit{Test} (\textit{scope=test}) a na modulech představují \textit{API} nižší vrstvy.
\end{itemize}

\begin{figure}
\begin{center}
\includegraphics[width=12cm]{figures/persistance_module}
\caption{Struktura perzistenční vrstvy}
\label{fig:des-persistence}
\end{center}
\end{figure}

\subsubsection{Vrstva datového přístupu}
\label{sec:des_data_access}
Vrstva datového přístupu je nejbližší vyšší vrstva perzistentní vrstvy, jejíž funkcionalitu rozšiřuje o kontrolu oprávnění uživatele na dotazovaná data\footnote{Grafové databáze typicky neumožňují definovat oprávnění jednotlivých uživatelů tak, jak je tomu například u \textit{RDBMS} databází.}. \textit{API} této vrstvy kopíruje \textit{API} perzistentní vrstvy, přičemž každou metodu dotazující data\footnote{Kontrola oprávnění není nutná ve všech případech. Existuje premisa, že pokud má uživatel oprávnění na instanci entity, pak musí mít nuntě oprávnění na všechny předky této instance v hiearchii grafu datových toků.} rozšiřuje o definici strategie, pomocí které dojde k validaci oprávnění uživatele. Z výsledků metod perzistentní vrstvy jsou tak v závislosti na zvolené strategii kontroly oprávnění případně odstraňována data, na která uživatel nemá oprávnění. Vrstva datového přístupu je \textit{uzamčená vrstva}.

\subsubsection{Algoritmy}
Další vrstva - \textit{Algoritmy} obsahuje funkcionalitu stávajícího modulu \textit{Connector}, respektive jeho částí \textit{Algorithm} a \textit{Traversal}. Jedná se o první vrstvu realizující byznys logiku aplikace (funkcionalita je popsána v kapitole \ref{sec:ana_connector}). Jedná se o \textit{odemčenou vrstvu}, může být tedy obcházena vyššími vrstvami v případě, že chtějí použít přímo \textit{API} vrstvy datového přístupu. Komponentami ve vyšších vrstvách serverové části aplikace jsou \textit{Merger, Viewer, Exporter a Public Api}.


\section{Orchestrace komponent Manta Flow}
\label{sec:des_orchestration}
Tato sekce se zabývá návrhem orchestrace jednotlivých aplikací, které tvoří \textit{Manta Flow}. Těmi jsou \textit{Server, Client, Configurator a Updater}. Prioritním požadavkem v rámci této sekce je \textit{F1}, tedy aby bylo možné pomocí \textit{Updateru} aktualizovat všechny aplikace, které jsou součástí \textit{Manta Flow}. Tomuto problému bude tedy věnována první část sekce. Pro zjednodušení budou v této sekci předpokládáno, že všechny tyto aplikace jsou nasazeny na jedno fyzické zařizení. Druhá část sekce bude věnována požadavkům \textit{F2 a F3} a bude tedy již počítat s možností nasazení jednotlivých komponent na více fyzických zařízení.

\subsection{Orchestrace na jednom zařízení}
\label{sec:des_orchestration_singlenode}
Primárním problémem v tomto případě je, jakým způsobem může být provedena aktualizace komponent \textit{Configurator} a \textit{Updater}. Aktualizace komponent \textit{Manta Flow Server a Client} v rámci jednoho fyzického zařízení je \textit{Updater} v současné době již bez problému schopen \cite{Gondek16}.

V případě komponenty \textit{Configurator} je problémový fakt, že obě aplikace (tedy \textit{Updater} i \textit{Configurator}) by měly být nasazeny na stejném aplikačním serveru (pro jednoduchost a efektivitu), přičemž k zaručení validní aktualizace aplikace musí být aplikační server restartován. To by vedlo k tomu, že v průběhu procesu aktualizace by byla zastavena i aplikace \textit{Configurator}, kterou by uživatel v tu dobu používal.

U komponty \textit{Updater} je problém analogický. Navíc je taková operace těžko proveditelná, protože aplikace nemůže používat své vlastní zdrjové soubory a konfigurace - v době běhu aplikace jsou používány a zamčeny.

V rámci návrhu řešení tohoto problému jsou definovány dvě nové komponenty:

\begin{itemize}
   \item{\textbf{Manta Flow Toolbox}}: Vzhledem k tomu, že aplikace \textit{Updater} a \textit{Configurator} fungují velmi podobným způsobem, mají stejnou vnitřní architekturu a měly by být nasazovány společně, mohou být sloučeny pro snadnější operace s nimi do jedné aplikace - pracovně pojmenované \textit{Manta Flow Toolbox}.\footnote{Přestože jsou aplikace sloučeny v jednu, měly by nadále fungovat jako samostatné moduly, které mezi sebou mají jasné rozhraní.}
   \item{\textbf{Update Batch}}: Vstupem komponenty \textit{Updater} při aktualizaci některé komopnenty \textit{Manta Flow} by měl být podle \cite{Gondek16} archiv obsahující nové, či upravené soubory (zdrojové, nebo konfigurační) aktualizované komponenty a soubor obsahující instrukce k aktualizaci pro \textit{Updater}.
   \textit{Update Batch} je zobecněním tohoto konceptu, přičemž by měl obsahovat archiv pro každou z komponent, která má být aktualizována (\textit{Server, Toolbox, Client}), a konzolovou aplikaci\footnote{Pro aktualizaci \textit{Toolboxu} může být použit \textit{shell} skript nebo jednoduchá \textit{Java} aplikace}, která provede aktualizaci \textit{Toolboxu} (popsáno sekvenčním diagramem \ref{fig:des-upd_toolbox}).
\end{itemize}

\textit{Server} a \textit{Client} mohou v rámci konkrétních nasazení obsahovat mnoho \textit{customizací} zdrojového kódu i konfiguračních souborů. Jejich aktualizace může být tedy v závislosti na množštví těchto \textit{customizací} velmi komplexní operace, která se neobejde bez účasti uživatele, který musí v některých případech zajistit správné sloučení některých souborů, které jsou při aktualizaci měněny. Na druhé straně u aplikací \textit{Updater} a \textit{Configurator} (respektive u aplikace \textit{Toolbox}) žádná \textit{customizace} nutné nejsou, jediná konfigurace, kterou obsahují, jsou odkazy na instance aplikací \textit{Server a Client}. Jejich aktualizaci je tak možné provést bez účasti uživatele skriptem, nebo jednoduchou aplikací. Spuštění této aplikace je prvním krokem aktualizace všech komponent a jeho součástí je aktualizace \textit{Toolboxu} a přesun vstupních dat pro aktualizace ostatních komponent na umístění, kde je \textit{Toolbox} očekává\footnote{Na rozdíl od popisu procesu aktualizace v \cite{Gondek16} už tak uživatel při aktualizaci ostatních komponent nemusí dohledávat manuálně vstupy pro aktualizace.}.

Tato architektura zajišťuje, že v případě instalace všech komponent na jedno fyzické zařízení je možné všechny instance jednoduše aktualizovat.

\subsection{Orchestrace po síti}
\label{sec:des_orchestration_multinode}
V případě, že jsou instance serverové a klientské části aplikace nasazené na různá fyzycká zařízení, architektura popsaná v předchozí kapitole není dostačující k tomu, aby mohly být obě tyto komponenty aktualizovány najednou.
Předpokládejme situaci, kdy je na jednom zařízení nasazen \textit{Manta Flow Server} a \textit{Toolbox}, a na druhém \textit{Manta Flow Client}. Je tak možné aktualizovat (a konfigurovat) \textit{Server} i \textit{Toolbox} (dle návrhu v předchozí kapitole), není ale možné pomocí \textit{Toolboxu} aktualizovat a konfigurovat \textit{Client}. Primárním problémem je, že \textit{Toolbox} nemá přístup na souborový systém zařízení, na kterém je \textit{Client} nasazen.

Tento problém by bylo teoreticky možné vyřešit pomocí některého z protokolů umožňující vzdálený přístup k souborovému systému (například \textit{FTP}). Tento přístup ale v současnosti není považovaný za standardní, ani za příliš bezpečný, není tak možné se spoléhat na to, že by byl v kontextu každého uživatele povolen. Nabízí se tak dnes již standardnější způsob síťové komunikace - protokol \textit{HTTPs}.

V případě, kdy je \textit{Toolbox} instalován společně se aktualizovanou komponentou na jednom zařízení, průběh aktualizace probíhá tak, jak je zachyceno na diagramu \ref{fig:des-upd_server}. K tomu, aby komunikace mezi oběmi komponentami mohla probíhat obdobným způsobem i při použití protokolu \textit{HTTPs}, je zapotřebí aby na obou stranách komunikace byly aplikace, které jsou v provozu nepřetržitě. Tomu neodpovídá současná architektura klientské částti aplikace, která je řádkovou aplikací spouštěnou \textit{ad-hoc}. Je tedy navržena úprava architektury aplikace \textit{Manta Flow Client} tak, aby reflektovala výše uvedené požadavky.

Jak je zřejmé z diagramu \ref{fig:des-upd_client}, který zobrazuje průběh aktualizace aplikace \textit{Manta Flow Client} přes \textit{HTTPs}, je nutné, aby se tato aplikace skládala ze dvou komponent:

\begin{itemize}
   \item{\textit{Dávková aplikace}}: Dávková \textit{Java} aplikace, zcela odpovídá aktuální podobě klientské části aplikace, včetně způsobu použití.
   \item{\textit{Webová aplikace}}: Webová aplikace bude obalovat dávkovou aplikaci a bude sloužit primárně pro komunikaci \textit{Manta Flow Toolbox}. Tato aplikace by měla být navržena striktně dle architektonického stylu \textit{REST} a měla by být pokud možno minimalistická co se týká vlastní konfigurace a poskytovaných funkcí. Vznikem této komponenty totiž vzniká problém s její aktualizací. Ten je řešitelný vygenerováním aktualizačního skriptu \textit{Toolboxem} (na základě rozhodnotí uživatele o případném sloučení konfigurací).
\end{itemize}

Diagram \ref{fig:ana-deployment-changed} ukazuje změna v nasazení jednotlivých aplikací.


\section{Možnosti horizontálního škálování aplikace}
\label{sec:des_scaling}
V kapitole \ref{sec:ana_scaling} je uvedeno, že \textit{Manta Flow} může mít v budoucnu problémy s množstvím dat, které zpracovává. Aplikace může narazit na mezní hranici kapacity metadatového úložiště, která je v současné době omezena především používáním \textit{embedded} databáze \textit{Persistit} jako podkladové vrstvy pro \textit{Titan}. Současně může být v některých kontextech problematický i výpočetní čas operací byznys logiky aplikace (viz sekce \ref{sec:ana_performance}), který při současné architektuře narůstá lineárně v závislosti na vstupních datech.

V této kapitole jsou popsány možnosti, jak může být stávající architektura aplikace \textit{Manta Flow} (nutně zahrnující změny popsané v kapitole \ref{sec:des_api} a volitelně zahnující změny popsané v kapitole \ref{sec:des_orchestration}) postupnými kroky upravována tak, aby průběžně vyhovovala zvyšujícím se požadavkům na horizontální škálování.


\subsection{Škálování grafové databáze}
\label{sec:des_scaling_db}
Použití \textit{embedded} databáze \textit{Persistit} je zjevně potenciální úzké hrdlo aplikace a jeho nahrazení horizontálně škálovatelnou grafovou databází je tak první změna architektury směrem k vyšší škálovatelnosti celé aplikace. Jsou identifikovány dva hlavní důvody, které vedou k tomu, že je používána \textit{embedded} databáze.

Prvním je výkon aplikace, který by při použití stávající architektury a jiné než \textit{embedded} databáze výrazně utrpěl - síťová komunikace představuje při komunikaci aplikace se vzdálenou databází vždy režii navíc, v případě \textit{RDBMS} databází (a většiny \textit{NoSQL} databází) je tato položka ale pro celkový výkon aplikace zanetbatelná. Důvodem je, že dotazy vykonávané těmito databázemi jsou často velmi komplexní a jejich vykonání trvá zpravidla déle, než síťová komunkace mezi aplikací a databází. U grafových databází tomu ale tak být nemusí, jazyky pro dotazování těchto databází stále prochází vývojem a ne vždy poskytují takové nástroje, aby pomocí nich mohly být realizovány všechny komplexní grafové algoritmy. To vede v některých případech k tomu, že granularita dotazů do grafové databáze je velmi vysoká (dotazy se v některých případech omezují pouze na okolní hrany/uzly) a čas vykonávání těchto dotazů je tak srovnatelný, nebo dokonce menší, než doba potřebná pro síťovou komunikaci mezi aplikací a grafovou databází.

Druhým omezením je závislost aplikace na dotazovacím jazyce \textit{Gremlin 2.x}, který je vzhledem k možnosti přímého dotazování grafové databáze pomocí \textit{PTM} transakcí (viz sekce \ref{sec:ana_visibility}) používán všemi moduly serverové části aplikace a je tak obtížně nahraditelný. Některé uvažované databáze, které by mohly případně aktuálně používaný \textit{Perzistit} nahradit, jazyk \textit{Gremlin} ve verzi \textit{2.x} nepodporují.

Změna architektury komponenty \textit{Connector} navržená v kapitole \ref{sec:des_api} oba tyto problémy řeší. Všechny dotazy do grafové databáze jsou definovány v nově navržené perzistentní vrstvě a jsou tak jednoduše upravitelné (včetně změny dotazovacího jazyk, kterým jsou realizovány) bez dopadů na ostatní komponenty (díky definovanému privátnímu \textit{API}). To je navrženo tak, aby byla ve vhodných případech snížena granularita dotazů a byl tak omezen problém režie síťové komunikace aplikace se vzdálenou databází. Příkladem jsou metody typu \textit{query}, které umožňují provádění komplexních dotazů na starně grafové databáze.

Díky tomu je otevřena možnost použití vzdálené grafové databáze, a je tedy možné \textbf{využít horizontální škálovatelnost grafových databází}, která je nativní vlastností většiny z nich. Je zřejmé, že lépší škálovatelnost budou vykazovat spíše komplexnější dotazy do grafové databáze, než dotazy atomické. Je proto možné, že pro maximalizaci využití horizontální škálovatelnosti grafové databáze bude nutné některé algoritmy, nebo jejich části, implementovat zcela pomocí komplexních grafových dotazů (například pomocí nadstavby jazyka \textit{Gremlin} - \textit{Pipes}). Obdobným způsobem, jako jsou navrženy metody typu \textit{query} mohou být navrženy metody pro průchody grafem datových toků, mohly by tak vznikat komplexní průchody, kde by algoritmus průchodu nebyl vázán ná konkrétní dotazovací jazyk, zároveň by ale byly tyto průchody vykonávány kompletně na straně grafové databáze.

Horizontálně škálovatelná grafová databáze by měla být dostatečným řešením pro potřeby aplikace \textit{Manta Flow} a zároveň se jedná o relativně neinvazivní změnu v architektuře aplikace. \textit{Manta Flow} je software dodávaný tzv. \textit{on premise}, uživatel tedy instaluje aplikaci na vlastní infrastrukturu a má nad ní plnou kontrolu. Uživatel si tak může sám rozhodnout, zda si vystačí s \textit{embedded} grafovou databází, nebo zda potřebuje horizontálně škálovatelnou grafovou databázi a na jaké infrastruktuře bude v tom případě nasazena. Díky navržené vícevrstvé architektuře není problém, aby aplikace podporovala obě možnosti a použití konkrétní implementace databázové (a perzistentní) vrstvy bylo konfigurovatelné.


\subsection{Škálování Java komponent aplikace}
\label{sec:des_scaling_java}
Dalším krokem pro zlepšení škálovatelnosti aplikace může být potenciálně horizontální škálování samotných \textit{Java} komponent aplikace. Protože tento krok již předpokládá použití horizontálně škálovatelné grafové databáze, je cílem v tomto případě již pouze zvýšení výkonu aplikace - schopnost analýzy zdrojových dat velkého objemu je již zaručena. Na elementární bázi to umožňuje i stávající architektura klient-server (v případě klientské části aplikace). Vzhledem k tomu, že klientská část aplikace obsahuje mimo samotné extrakce dat (databázových slovníků, zdrojových kódů, transformací, atd.) ze zdrojových systémů také logiku pro jejich analýzu, může několik instancí klientské části aplikace výkon zvýšit významně.\footnote{Paralelní instance klientské části aplikace ale není možné (či efektivní) použít ve všech případech, v případě silně propojených zdrojových systémů by musely jednotlivé instance sdílet témeř všechna data.} Teoreticky by také bylo možné paralelní nasazení několika serverových komponent, to by si ale vyžádalo úpravu některých algoritmů a především zavedení dalšího sdíleného zdroje (krom vstupů a grafové databáze), který by sloužil pro synchronizaci zámků v databázi (viz kapitola \ref{sec:des_transactions}). Pro nadměrnou komplexitu a nejistou efiktivitu tak tento přístup není možné doporučit.

Efektivněji škálovatelná by serverová část aplikace byla, pokud by její architektura byla upravena postupným vyčleňováním jednotlivých služeb podle architektonického stylu \textit{microservices}. Vzhledem ke komplexní a specifické infrastruktuře, kterou takové řešení vyžaduje, by ale taková architektura nebyla vhodná pro software dodávaný jako \textit{on premise}. Přechod na software \textit{on demand} je ale rozhodnutí zahrnující více faktorů než jen architektonická omezení aplikace a v současné době není vzhledem k prototypu uživatele aplikace možný.

\chapter{Implementace prototypu}
Součástí práce je prototypová implementace architektury navržené v kapitole \ref{sec:des_api}, konkrétně komponent \textit{Doménový model, Databázová vrstva, Perzistentní vrstva a Vrstva datového přístupu}. Prototypová implementace také obsahuje část byznys logiky vyšších vrstev aplikace, na jejímž základě je provedena validace navržené architektury a navrženého \textit{API} perzistentní vrstvy.  Na obrázku \ref{fig:poc_components} je \textit{UML} diagram komponent prototypové implementace. V této kapitole jsou blíže popsány důležité či nestandardní části implementace a je diskutována validnost vytvořeného návrhu.

%Domain
\section{Doménový model}
Doménový model definuje typy entit, se kterými může aplikace pracovat a možné parametry těchto entit. Prakticky tak definuje také schéma (datový model) grafové databáze, přestože ta nutně nemusí koncept schémat podporovat. Hlavní entity doménového modelu jsou popsány \textit{\nomExpl{BDM}{Business Domain Model} UML diagramem} \ref{fig:impl-domain} (cílem diagramu je popsat entity a vztahy mezi nimy, ne konkrétní implementaci). Při porovnání s datovým modelem metadatového úložiště je patrné, že:

\begin{itemize}
   \item Pro každý typ uzlu, který je součástí datového modelu existuje ekvivalent v doménovém modelu (s adekvátně definovanými parametry). Výjimku tvoří umělé typy uzlů, tedy \textit{REVISION\_ROOT, SOURCE\_ROOT a SUPER\_ROOT}.
   \item Doménový model neobsahuje tzv. \textit{řídící hrany} datového modelu, tedy hrany, které tvoří hiearchickou strukturu uzlů. Jediným netechnickým parametrem těchto hran je interval platnosti uzlů. Ten je ale spíše vlastností uzlů samotných, nikoliv jejich řídících hran (jeho umístění na řídící hrany v datovém modelu je důsledek optimalizace výkonu aplikace). Nic tedy nebrání tomu, aby byl doménový model zjednodušen a tyto hrany z něj odebrány.
   \item Doménový model obsahuje ekvivalenty hran typu \textit{DIRECT, FILTER a MAPS\_TO} datového modelu. Tyto hrany mají vlastní interval platnosti, který je sice omezen intervaly platnosti uzlů, které spojují, ale může se od těchto intervalů lišit. Hrany také obsahují další parametry podstatné pro analýzu datových toků. Entita doménového modelu \textit{Flow} zahrnuje hrany datového modelu \textit{DIRECT i FILTER}.
\end{itemize}


\section{Databázová vrstva}
Databázová vrstva obsahuje dvě stežejní třídy (které musí obsahovat vždy) a to jsou implementace tříd \texttt{org.springframework.transaction.support.ResourceTransactionManager} a \texttt{com.tinkerpop.blueprints.Graph}. Tato rozhraní \textit{de-facto} definují \textit{API} databázové vrstvy. \textit{Transaction manager} musí být implementován, aby mohly vyšší vrstvy využívat deklarativní transakční model, který je pro návrh celé vícevrstvé architektury velmi důležitý. \textit{Graph} zpřístupňuje data uložená v grafové databázi dotazovacímu jazyku \textit{Gremlin 2.x}. V tomto případě nevadí, že je \textit{API} vrstvy závislé na konkrétním programovacím jazyku - implementace nejbližší vyšší vrstvy (perzistentní vrstvy) je (a vždy bude) z velké části tvořena právě tímto dotazovacím jazykem.

%Perzistence
\section{Perzistentní vrstva}
Implementace perzistentní vrstvy je tvořena především implementací \textit{repository} objektů definovaných v \textit{API} vrstvy (popsáno v kapitole \ref{sec:des_persistence}). Dokumentace celého API perzistentní vrstvy je dostupná na přiloženém CD (viz obsah CD v příloze \ref{apx:cd}).

\subsection{Vrstva \textit{mapperů}}
Modul obsahující implementaci perzistentní vrstvy obsahuje další vnitřní vrstvu tzv. \textit{mapperů} - objektů provádějících mapování vrcholů a hran grafové databáze (reprezentovaných \textit{Gremlin} třídami \textit{Vertex a Edge}) na objekty doménového modelu.  Pro účely prototypu byla zvolena implementace pro (aplikací aktuálně používaný) jazyk \textit{Gremlin 2.x}.

\subsection{Dotazy do metadatového úložiště}
Jednoduché dotazy jsou implementovány nativními dotazy jazyka \textit{Gremlin}. Příklad \ref{exa:mapper} ukazuje mapování entity \textit{Flow} pomocí jazyka \textit{Gremlin 2.x}.

\lstinputlisting[language=Java, caption=Mapování entity \textit{Flow} (implementace pomocí \textit{Gremlin 2.x}), label=exa:mapper]{code/clear_gremlin.java}

Pro větší efektivitu dotazů byly v některých případech nativní \textit{Gremlin} dotazy nahrazeny dotazy pomocí další součástí \textit{TinkerPop} projektu - \textit{Pipes}\footnote{\url{https://github.com/tinkerpop/pipes/wiki}}. \textit{Pipes} slouží právě jako nástroj pro komplexnější průchody grafem pro jazyk \textit{Gremlin 2.x}\footnote{\textit{Gremlin 3.x} nahrazuje \textit{Pipes} navazujícím řešením s názvem \textit{Graph Traversal}}. Průchodu grafu pomocí \textit{Pipes} je ukázán na příkladu \ref{exa:pipes}, kde uvedený kód nalzene instanci entity \textit{Node} podle kvalifikovaného jména (jedná se tedy o průchod víceúrovňovou hierchií grafu datových toků).

\lstinputlisting[language=Java, caption=Nalezení uzlu dle kvalifikovaného jména (implementace pomocí \textit{Pipes}), label=exa:pipes]{code/pipes.java}

\subsection{Rozhraní pro \textit{query} metody}
V kapitole \ref{sec:des_persistence} je popsán návrh \textit{query} metod. Ty by měly sloužit k obecnému dotazování dat z metadatového úložiště podle typu entity. Jedním z důvodů návrhu těchto metod je snaha o umožnění tvorby komplexnějších dotazů, které by měly být efektivnější, než kompozice atomických dotazů. Výsledkem toho je, že vstupem pro tyto metody jsou komplexní do sebe zanořené dotazovací objekty. Aby byl tento koncept reálně použitelný, bylo nutné vytvořit samostatné rozhraní pro tvorbu těchto objektů - jinak by tvorba dotazovacích objektů byla příliš komplikovaná, nutně by navíc musela obsahovat některé perzistenční detaily, od kterých by měl být vývojář při používání těchto metod odstíněn.
Návrh byl inspirován návrhovým vzorem \textit{Builder Pattern} a stylem \textit{Fluent Interface} \cite{Fowler05}. Ukázka \ref{exa:fluent_api} obsahuje dotaz na všechny \textit{nullable} sloupce ze specifické tabulky, schématu a databáze. Pomocí stejného návrhu by bylo možné provádět průchody grafem datavých toků.

\lstinputlisting[language=Java, caption=Fluent interface query metod, label=exa:fluent_api]{code/fluent_api.java}

%Data-access
\section{Vrstva datového přístupu}
\textit{API} této vrstvy v zásadě kopíruje \textit{API} perzistentní vrstvy.  U metod, u kterých musí být ověřována práva uživatele na dotazovaná data, je ale navíc přidán parametr reprezentující strategii, pomocí které budou data ověřována. Asi tedy nepřekvapí, že je vrstva implementována pomocí návrhovému vzoru \textit{Strategy} - aktuálně existují dvě strategie (v budoucnu může být ale tento seznam rozšířen):

\begin{itemize}
   \item{\textit{Oprávnění na základě pohledů a rolí}}: Je zaveden nový termín \textit{pohled}. Ten reprezentuje část dat uložených v metadatovém úložišti - každý pohled se skládá ze sady zahrnutých a vyloučených objektů ve smyslů hiearchie grafu datových toků (entity \textit{Node} a \textit{Resource}). Pohledů může být teoreticky neomezené množství a mohou se navzájem překrývat. Uživatelům jsou pak přidělována oprávnění na jednotlivé pohledy na základě jejich \textit{LDAP\footnote{Lightwight Directory Access Protocol}} rolí.

   \item{\textit{Neomezená oprávnění}}: Existují operace, u kterých není za žádných okolností žádoucí oprávnění kontrolovat. Například kompletní exporty metadatového úložiště. Proto existuje strategie přidělující všem uživatelům neomezená oprávnění.
\end{itemize}

Pro implementaci strategie kontroly oprávnění na základě pohledů je navíc za účelem rozdělení zodpovědností použit návrhový vzor \textit{Chain of responsibility} (součástí implementace je také vlastní \textit{cache} a další logika). Tato implmenetace je pospána diagramem tříd \ref{fig:impl-permission}.


%validation
\section{Validace a testování}
Funkcionalita všech implementovaných modulů byla pokryta jednotkovými testy. Tabulka \ref{tab:coverage} obsahuje údaje o pokrytí implementace jednotkovými testy.

\begin{table}[h!]
\begin{center}
\centering
\caption{Pokrytí prototypové impolementace návrhu jednotkovými testy}
\label{tab:coverage}
\begin{tabular}{|p{4cm}|p{4cm}|p{4cm}|}
	\hline
    Třídy & Metody & Řádky \\ \hline
	 86\% (150/174) &	73\% (726/994) &	78\% (3476/4438) \\ \hline
 \end{tabular}
 \end{center}
 \end{table}

V rámci validace návrhu a prototypové implementace vícevrstvé architektury byly pomocí vytvořeného prototypu implementovány některé operace vyšších vrstev aplikace, konkrétně:

\begin{itemize}
   \item{\textit{Merge}}: Byla implementována operace \textit{merge}, tedy stěžejní část operace updatu metadatového úložiště. Operace je blíže popsána v kapitole \ref{sec:ana_merger}.
   \item{\textit{Flow Algorithm}}: Byl také implementován základní algoritmus pro hledání datových toků. Tento algoritmus je součástí mnoha sloužitějších algoritmů, které \textit{Manta Flow} používá.
\end{itemize}

Pro oba zmíněné algoritmy je v rámci stávající implementace definována řada jednotkových testů. Základním předpokladem validace vytvořeného prototypu je, že podaří-li se pomocí nově navržené architektury implementovat obě tyto funkcionality a budou-li nadále úspěšně procházet všechny testy pokrývající tyto funkcionality, podařilo se navrhnout architekturu a implementovat prototyp vyhovující zadání práce.

To se podařilo, lze tedy prohlásit, že \textbf{prototyp je validní}.

Další typy testů (např. \textit{performance} testy) nebyly provedeny, protože v kontextu této práce nedávají smysl. Cílem validace je ověřit vhodnost navržené architektury pro konkrétní funkcionality \textit{Manta Flow}. Vytvořená implementace (byť rozsáhlá - více než 22 tisíc \textit{LOC}) je navíc pouze prototypem, neobsahuje tak všechny optimalizace výkonu, které zahrnuje stávající aplikace.

\chapter{Závěr}
Cílem této práce bylo navržení vícevrstvé architektury a příslušných \textit{API} pro práci s grafovou databází realizující metadatové úložiště aplikace \textit{Manta Flow}. Tato architektura byla navržena, byla vytvořena její prototypová implementace a ta byla otestována. Prototypová implementace navržené architektury zahrnující také dvě stěžejní operace byznys logiky aplikace prokazuje, že vytvořený návrh je pro práci s metadatovým úložištěm aplikace \textit{Manta Flow} vhodný. Navržená architektura striktně odděluje byznys logiku aplikace a perzistentní logiku pro ukládání metadat do grafové databáze pomocí vytvořeného \textit{API} perzistentní vrstvy, a řeší tak řadu současných architektonických omezení aplikace.

% modularity
Nejpodstatnějším důsledkem navržené architektury je umožnění změny grafové databáze (i změny jazyka dotazujícího grafovou databázi) realizující metadatové úložiště bez dopadů na vyšší vrstvy aplikace. Tento požadavek měl největší prioritu, protože aktuálně použiváná grafová databáze \textit{Titan} již není nadále vyvýjena a brzy nebude ani podporována. Zároveň dotazovací jazyk \textit{Gremlin 2.x}, který byl hojně používán i v implementacích byznys logiky aplikace nepodporuje všechny grafové databáze, které by mohly být potenciálně vhodné pro realizaci metadatového úložiště \textit{Manta Flow}.

% transactinos and domain model
Aby bylo zajištěno správné používání navrženého \textit{API}, musely být eliminovány možnosti jeho obcházení. Ty spočívaly především v používaném transakčním modelu, který umožňoval v kombinaci s používanými entitami doménového modelu \textit{TinkerPop Blueprints} přímé dotazy do grafové databáze z téměř jakékoliv části serverové části aplikace. Byl tak navržen nový doménový model aplikace a byl zaveden deklarativní transakční model. Ten zaručuje možnost propagace transakcí do vyšších vrstev aplikace bez toho, aby tak byl z těchto vrstev umožněn přímý přístup do grafové databáze. Pro použití deklarativního transakčního modelu musí být splněny (v závislosti na frameworku, který jej implementuje) jisté předpoklady. U grafových databází zatím neexistují standardní implementace používající deklarativní transakční model, byly tak vytvořeny dvě \textit{PoC} implementace ověřující možnost použití tohoto transakčního modelu v rámci frameworku \textit{Spring} a s využitím dotazovacích jazyků \textit{Gremlin 2.x} a \textit{Gremlin 3.x}. Obě implementace možnost použití deklarativního transakčního modelu potvrdily.

% scaling
Definované \textit{API} perzistentní vrstvy navržené architektury také mění granularitu dotazů do metadatového úložiště. Stávající implementace používá často atomické dotazy do grafové databáze a \textit{de-facto} tím kvůli minimalizaci režie dotazů vynucuje použití \textit{embedded} databáze jako podkladové vrstvy pro grafovou databázi \textit{Titan}. Navržené \textit{API} umožňuje používání komplexnějších dotazů do metadatového úložiště a v zásadě tak umožňuje použití vzdálené grafové databáze. Tím jsou otevřeny nové možnosti pro horizontální škálování aplikace. V práci je navrženo několik návazných úprav architektury aplikace pro dosažení horizontálního škálování, přičemž první a potenciálně nejefektivnější z nich je použití vzdálené horizontálně škálovatelné grafové databáze pro realizaci metadatového úložiště.

% orchestration
Součástí celého řešení \textit{Manta Flow} se také nově staly aplikace \textit{Configurator} a \textit{Updater}, které zatím ale nejsou ukotveny v architektuře aplikace, která se navíc mění výše popsaným způsobem. Bylo tak nutné vytvořit návrh orchestrace všech aplikací tvořících celé řešení \textit{Manta Flow} a to v případě nasazení všech aplikace na jedno zařízení, nebo na několik různých zařízení. Tento návrh má za následky další úpravy architektury aplikací \textit{Manta Flow Client, Updater a Configurator}. V práci jsou tyto návrhy úprav architektury jednotlivých aplikací popsány a zdůvodněny.

\section{Možnosti dalšího rozvoje}

V diplomové práci bylo navrženo několik změn architektury aplikace \textit{Manta Flow}. Pro část z nich, konkrétně pro vícevrstvou architekturu navrženou pro přístup do metadatového úložiště, byla také vytvořena prototypová implementace ověřující vhodnost vytvořeného návrhu. Nejdůležitější návaznou činností na tuto práci tak je posouzení návrhů na úpravy architektury aplikace a jejich případná implementace do produkční verze aplikace.

V průběhu tvorby diplomové práce bylo také identifikováno několik oblastí, které by mohly být dále zkoumány a rozvíjeny.

% zámky & paralelism
Kvůli neschopnosti aplikace provádět paralelní vkládání objektů do grafové databáze byl navržen nový algoritmus pro zamykání objektů v databázi. Díky němu je nyní teoreticky možné řádově zrychlit několikahodinový proces aktualizace metadatového úložiště. Je však nutné upravit algoritmy, které jsou součástí tohoto procesu, tak, aby byly schopné nový systém zámků efektivně využít.

% query a traverse metody
Součástí navrženého \textit{API} perzistentní vrstvy jsou \textit{query} metody, které slouží pro tvorbu komplexních dotazů do metadatového úložiště. Byly uvedeny argumenty, proč je výhodnější sestavování komplexních dotazů, které jsou jako celek vykonávány grafovou databází, než řetězení dotazů atomických. Mohlo by proto být výhodné rozhraní \textit{query} metod dále rozšířit, aktuálně totiž nepokrývá všechny typy dotazů. Analogicky by také mohl být definován nový typ metod \textit{API} - \textit{traverse}, který by pomocí obdobně strukturovaných dotazů prováděl průchody grafovou databází. Algoritmy, které jsou součástí byznys vrstvy aplikace by tak mohly provádět řádově méně dotazů do grafové databáze, čímž by bylo možné výrazně snížit režii těchto dotazů.

% microservices
V sekce návrhu architektury týkající se škálovatelnosti aplikace byla mimo jiné diskutována možnost úpravy navržené architektury postupnými kroky na architekturu \textit{microservices}, která by umožnila lepší horizontální škálování aplikace. Vzhledem k nevhodnosti této architektury pro aktuální model nasazení aplikace byla tato možnost v práci označena jako nevhodná pro aplikaci \textit{Manta Flow}. Je ale možné, že kvůli zvyšujícím se nárokům na objem zpracovávaných dat bude nutné ji v budoucnu opět zvážit.




%%%%%%%%%%%%%%%%%%%%%%%%%%
% Seznam literatury je v samostatnem souboru reference.bib.
\bibliographystyle{csplainnat}
{
\def\CS{$\cal C\kern-0.1667em\lower.5ex\hbox{$\cal S$}\kern-0.075em $}
\bibliography{reference}
}

%%%%%%%%%%%%%%%%%%%%%%%%%%
% vše co následuje bude uvedeno v přílohách
\appendix

\chapter{Seznam zkratek}
\printnomenclature
\label{apx:zkratky}

\textbf{ACID} - \textit{Atomicity, Consistency, Isolation, Durability}

\textbf{API} - \textit{Application Programming Interface}

\textbf{BASE} - \textit{Basically Available, Soft state, Eventual consistency}

\textbf{BFS} - \textit{Breadth First Search}

\textbf{CRUD} - \textit{Create, Read, Update, Delete}

\textbf{CSV} - \textit{Comma Separated Variable}

\textbf{DDL} - \textit{Data Definition Language}

\textbf{DFS} - \textit{Depth First Seacrh}

\textbf{DSL} - \textit{Domain Specific Language}

\textbf{DTM} - \textit{Declarative Transaction Model}

\textbf{ETL} - \textit{Extract, Transform, Load}

\textbf{FaaS} - \textit{Functino as a Service}

\textbf{FTP} - \textit{File Transfer Protocol}

\textbf{GUI} - \textit{Graphical User Interface}

\textbf{HTTP} - \textit{Hypertext Transfer Protocol}

\textbf{HTTPS} - \textit{Hypertext Transfer Protocol Secure}

\textbf{IaaS} - \textit{Infrastructure as a Service}

\textbf{JSON} - \textit{JavaScript Object Notation}

\textbf{JVM} - \textit{Java Virtual Machine}

\textbf{LOC} - \textit{Lines of Code}

\textbf{MVCC} - \textit{Multi-Version Concurrency Control}

\textbf{NoSQL} - \textit{Not only SQL}

\textbf{OGM} - \textit{Object Graph Mapper}

\textbf{OLAP} - \textit{Online Analytical Processing}

\textbf{ORM} - \textit{Object Relational Mapper}

\textbf{PaaS} - \textit{Platform as a Service}

\textbf{PoC} - \textit{Proof of Concept}

\textbf{POJO} - \textit{Plain Old Java Object}

\textbf{PTM} - \textit{Programatic Transaction Model}

\textbf{RDBMS} - \textit{Relational Database Management System}

\textbf{REST} - \textit{Representational State Transfer}

\textbf{RPC} - \textit{Remote Procedure Call}

\textbf{SaaP} - \textit{Software as a Product}

\textbf{SaaS} - \textit{Software as a Service}

\textbf{SOA} - \textit{Service Oriented Architectures}

\textbf{SOAP} - \textit{Simple Object Access Protocol}

\textbf{SPI} - \textit{Service Provider Interface}

\textbf{SQL} - \textit{Structured Query Language}

\textbf{URI} - \textit{Uniform Resource Identifier}

\textbf{WSDL} - \textit{Web Services Description Language}

\textbf{XML} - \textit{Extensible Markup Language}


\chapter{Diagramy}

\begin{figure}
\begin{center}
\includegraphics[width=11cm]{figures/flow_comp_vertical}
\caption{Stávající architektura \textit{Manta Flow}}
\label{fig:ana-flow-comp}
\end{center}
\end{figure}

\begin{figure}
\begin{center}
\includegraphics[width=14cm]{figures/flow_seq}
\caption{Interakce mezi \textit{klientskou} a \textit{serverovou} částí \textit{Manta Flow}}
\label{fig:ana-flow-seq}
\end{center}
\end{figure}

\begin{figure}
\begin{center}
\includegraphics[width=14cm]{figures/deployment_1}
\caption{Aktuální orchestrace aplikací \textit{Manta Flow Server, Client, Updater a Configurator}}
\label{fig:ana-deployment}
\end{center}
\end{figure}

\begin{figure}
\begin{center}
\includegraphics[width=1\linewidth]{figures/update_toolbox_seq}
\caption{První krok aktualizace všech komponent - aktualizace \textit{Manta Flow Toolbox}}
\label{fig:des-upd_toolbox}
\end{center}
\end{figure}

\begin{figure}
\begin{center}
\includegraphics[width=1\linewidth]{figures/update_server_seq}
\caption{Standardní aktualizace komponent na jednom zařízení}
\label{fig:des-upd_server}
\end{center}
\end{figure}

\begin{figure}
\begin{center}
\includegraphics[width=1\linewidth]{figures/update_client_seq}
\caption{Aktualizace komponenty \textit{Manta Flow Client} přes \textit{HTTPS}}
\label{fig:des-upd_client}
\end{center}
\end{figure}

\begin{figure}
\begin{center}
\includegraphics[width=14cm]{figures/deployment_changed}
\caption{Upravená orchestrace aplikací \textit{Manta Flow Server, Client a Toolbox}}
\label{fig:ana-deployment-changed}
\end{center}
\end{figure}

\begin{figure}
\begin{center}
\includegraphics[width=14cm]{figures/domain_model}
\caption{Doménový model}
\label{fig:impl-domain}
\end{center}
\end{figure}

\begin{figure}
\begin{center}
\includegraphics[width=1.3\linewidth, angle=90]{figures/permission-class}
\caption{Implementace kontroly oprávnění}
\label{fig:impl-permission}
\end{center}
\end{figure}

\begin{figure}
\begin{center}
\includegraphics[width=14cm]{figures/modules}
\caption{Diagram komponent prototypové implementace}
\label{fig:poc_components}
\end{center}
\end{figure}

\chapter{Obsah přiloženého CD}
\label{apx:cd}

\begin{itemize}
   \item{\textit{prototypova\_implementace}} (spustitený soubor vytvořené prototypové implementace)
   \item{\textit{DP\_Moravec\_Jakub\_2018}} (text diplomové práce ve formátu \textit{PDF})
   \item{\textit{dokumentace}} (dokumentace prototypové implementace)
      \begin{itemize}
         \item{\textit{api\_javadoc}} (\textit{javadoc} dokumentace navrženého \textit{API} perzistentní vrstvy)
         \item{\textit{diagramy}} (diagramy obsažené v diplomové práci)
      \end{itemize}
   \item{\textit{zdrojove\_kody}}
      \begin{itemize}
         \item{\textit{prototyp}} (zdrojové kódy prototypové implementace navržené architektury)
         \item{\textit{PoC}} (zdrojové kódy dvou \textit{PoC} implementací provedených v rámci sekce \ref{sec:des_transactions})
         \item{\textit{diplomova\_prace}} (zdrojové kódy diplomové práce ve formátu \textit{TEX})
      \end{itemize}
   \item{\textit{knihovny}} (privátní knihovny \textit{Manta FLow} nutné k sestavení prototypové implementace)
   \item{\textit{README.MD}} (soubor definující požadavky k sestavení, spuštění a korektní funkčnosti prototypové implementace)
\end{itemize}


\end{document}
